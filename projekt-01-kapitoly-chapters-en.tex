\chapter{Introduction}

Randomisation is essential to research areas such as \textit{probabilistic programming},~dependability (system components with uncertainty),~distributed computing (symmetry breaking),~and planning (unknown and noisy environments).
Probabilistic programs are powerful modelling apparatus of systems containing probabilistic uncertainty.
Systems with unreliable and unpredictable behaviour require the~use of such a~mathematical apparatus established on probability theory.
Designing such systems exhibiting a~desirable behaviour \,--\, e.g. selecting the~optimal power management strategy or a~network protocol increasing the~packet throughput \,--\, is challenging for reasoning over multiple alternative designs.
Their applications cover a~broad range of research areas,~including,~e.g. analysis of (quantitative) software product lines~\cite{spl1,spl2},~strategy synthesis in planning under partial observability~\cite{pomdp1,pomdp2},~or design of communication protocols~\cite{herman1,herman2}.

A~set of declarative temporal constraints often expresses the~efficiency and correctness of the~probabilistic programs.
The~model checkers for probabilistic systems,~such as \storm{}~\cite{STORM} or \prism{}~\cite{KNP11},~provide automated verification of such constraints.
However,~these probabilistic model checkers require a~fixed model or a~fixed program,~contrary to their usage requirements when modelling probabilistic programs.
Developers need to verify the~system designs as early as possible at the~developing process to maintain its costs tractable and as best as possible.
System designs are prevailingly incomplete at the~initial development phases because,~in most cases,~there are no known all system specifications or intentionally left out potentially.
These undefined system specifications are called \textit{holes},~and they can,~e.g.,~reflect an~unspecified component for wireless specification or a~partially implemented controller.
The~synthesis's primary purpose is through analysis reveals a~concrete subsystem with fully-defined behaviour and eventually reveals optimal designs when they are requested.
A~vital aspect of the design cycle is design space explorations,~i.e. exploring all possible designs.
When considering a~Markov chain (MC) as the~mathematical apparatus of a~probabilistic program,~then the~design space represents a~family of such chains,~and the~synthesis task is to find the~one that satisfies a~given specifications.

A~\textit{one-by-one} approach can naively solve the~synthesis problem by analysing all unique designs~\cite{spl3,onebyone}.
On the contrary,~the whole design space can also be modelled as a~single \textit{all-in-one} Markov decision process~\cite{spl3,allinone}.
However,~enumerating all members of design space (realisations) is unfeasible due to its combinatorial explosion, and the size of such all-in-one MDP is proportional to the number of candidate designs.
Unfortunately,~the~double state-space explosion problem renders both of these approaches infeasible for large families.
Other approaches consider evolutionary search algorithms for the~synthesis of software systems~\cite{spl2}.
These methods remain incomplete and cannot efficiently solve more challenging systems,~e.g.,~which design satisfies the specification \textit{optimally}.

In this work,~we will focus on a~complete state-of-the-art approach for the~synthesis of probabilistic programs.
This approach was first introduced in \textit{Andriushchenko} master’s thesis~\cite{roman-DP}, followed~by its improvement in~\cite{tacas21}.
It combines two sophisticated methods providing an~analysis of whole design sub-families at once.
The~first method analyses each design from a~given sub-family individually and constructs critical sub-systems of counter-examples to prune all designs behaving incorrectly \,--\, the~so-called \textit{counter-example guided inductive synthesis} (CEGIS)~\cite{cegis}.
The~second method,~called abstraction refinement (AR)~\cite{cegar},~immediately analyses the~entire design space and refines it into design sub-families when the~analysis yields inconclusive results.
Both of these methods have shown convincing results,~although each faces certain limitations.
They are incomparable because one method can be more suitable for specific probabilistic programs classes and conversely.
The~approach presented in this work integrates both these methods and it manages to significantly outperform them,~sometimes by a~margin of orders of magnitude.

All these presented methods consider \textit{topology synthesis} task assuming a~finite set of parameters that affect the~model topology,~where the~individual parameters represent the~undefined system specifications.
This task focuses on Markov chains families having different topologies of the~state space and,~as a~consequence,~different sets of reachable states.
However,~another area of synthesis tasks considers a~Markov chain with fixed topology but undefined transition probabilities.
This area has been discussed by approaches of model repairing~\cite{model-repair-1,pathak-et-al-nfm-2015} and techniques for \textit{parameter synthesis}~\cite{ceska2014robustness,Quatmann2016}.
When modelling real-world systems,~combining the two tasks can quickly occur,~but the~support to solve such a~\textit{combined synthesis} task does not exist.

\subsubsection*{Key Contributions.}
This thesis considers a~novel integrated method introduced in the~previous works~\cite{roman-DP,tacas21} as a~base stone.
Initially,~this method was designed only for feasibility synthesis task with one specification.
However,~probabilistic programs often have to satisfy specifications expressed as a~conjunction of several temporal logic constraints.
Therefore, we designed an~extension of this method to support \textit{multi-property specifications} and \textit{optimal synthesis} task.
The~designed extension for multi-properties is performed in the~same loop as the~origin single-property synthesis, with~necessary modifications: \textit{AR} need to analyse satisfiable specifications within inference sub-families no longer,~and \textit{CEGIS} analyses each specification individually and constructs counter-examples whenever a~given specification is unsatisfiable.
Consequently,~the~novel integrated method inherits the~benefits of \textit{AR} and \textit{CEGIS} in its favour also at multi-property synthesis.
\textit{Optimal synthesis} is a~particular instance of \textit{multi-property synthesis},~and it can find an application in various domains.
In particular,~specification set includes so-called violation property representing the currently optimal solution,~and its threshold is updated whenever a~new optimal solution is found.
Moreover,~we designed support for the~relaxed variant of the~optimal synthesis,~so-called $\varepsilon$-optimal synthesis,~which is in most cases even faster.
We evaluate designed extensions on an~extensive set of real-world case studies. 
We confirm the~results of a~novel approach presented in the~previous works and found the~following conclusions relating to these extensions.
A~novel integrated method is orders of magnitude faster than one-by-one enumeration when analysing the~single property.
A~multi-property synthesis slows down both approaches,~although the~novel method slow down is almost negligible.
An~optimal synthesis slows down only the~integrated approach,~yet it is still incomparably faster than enumeration.
The~assumption that $\varepsilon$-optimal synthesis can significantly speed up the~whole optimal synthesis process was also confirmed.

\todo{Parameter synthesis ...}


\subsubsection*{Structure of this paper.}
In Chapter~\ref{chap:synthesis},~we formulate a~probabilistic synthesis problem and introduce a~state-of-the-art novel integrated method based on two modern approaches \textit{CEGIS} and \textit{AR}.
Chapter~\ref{chap:advanced} develops vital ideas associated with integrating the \textit{multi-property} synthesis and \textit{optimal} synthesis within the presented integrated method.
Then, in Chapter~\ref{chap:combined}, we develop critical ideas associated with integrating the combined synthesis \,--\, consists of topology and parameter synthesis \,--\, within the considered method.
Chapter~\ref{chap:paynt} introduces a~new tool called \toolname{} and its architecture,~which implemented the~presented methods.
Chapter~\ref{chap:experiments} evaluates our solutions on a~broad range set of real-world case studies and compares them with the~baseline enumeration approach.
Finally,~Chapter~\ref{chap:conclusion} closes this thesis with the~notes
and issues that can serve as a~baseline point for the~follow-up research and potential improvement of designed solutions.

\chapter{Synthesis of Probabilistic Programs}\label{chap:synthesis}

\section{Problem Statement}

\section{Synthesis Methods}
Existing methods for a~probabilistic programs synthesis can be separated into two orthogonal classes.
The~first class involves complete methods that prove the~non-existence,~or potentially optimally,~of the~given probabilistic program.
On the~contrary,~incomplete methods handling various evolutionary techniques and intelligent search strategies form the~second one.
However,~these incomplete methods cannot treat unfeasible and optimal synthesis problems compared to the~complete methods.
Therefore, we focus on the~complete state-of-the-art methods even though the~incomplete methods provide a~valuable flexibility level.
As a~reference and baseline method,~we consider a~\textit{one-by-one} approach that enumerates through each design space member~\cite{onebyone}.
An~explosion of the~design space renders this technique unfeasible for large families,~necessitating using advanced approaches utilising the~arbitrary structure of~the Markov chain family.


\chapter{Advanced Methods for Probabilistic Synthesis}\label{chap:advanced}


\chapter{Combined Probabilistic Synthesis}\label{chap:combined}


\chapter{Tool Architecture}\label{chap:paynt}


\chapter{Experimental Evaluation}\label{chap:experiments}


\chapter{Final Considerations}\label{chap:conclusion}


\section{Future Research}
\section{Conclusions}
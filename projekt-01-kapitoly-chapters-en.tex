\chapter{Introduction}

Randomisation is essential to research areas such as \textit{probabilistic programming},~dependability (system components with uncertainty),~distributed computing (symmetry breaking),~and planning (unknown and noisy environments).
Probabilistic programs are powerful modelling apparatus of systems containing probabilistic uncertainty.
Systems with unreliable and unpredictable behaviour require the~use of such a~mathematical apparatus established on probability theory.
Designing such systems exhibiting a~desirable behaviour \,--\, e.g. selecting the~optimal power management strategy or a~network protocol increasing the~packet throughput \,--\, is challenging for reasoning over multiple alternative designs.
Their applications cover a~broad range of research areas,~including,~e.g. analysis of (quantitative) software product lines~\cite{spl1,spl2},~strategy synthesis in planning under partial observability~\cite{pomdp1,pomdp2},~or design of communication protocols~\cite{herman1,herman2}.

A~set of declarative temporal constraints often expresses the~efficiency and correctness of the~probabilistic programs.
The~model checkers for probabilistic systems,~such as \storm{}~\cite{STORM} or \prism{}~\cite{KNP11},~provide automated verification of such constraints.
However,~these probabilistic model checkers require a~fixed model or a~fixed program,~contrary to their usage requirements when modelling probabilistic programs.
Developers need to verify the~system designs as early as possible at the~developing process to maintain its costs tractable and as best as possible.
System designs are prevailingly incomplete at the~initial development phases because,~in most cases,~there are no known all system specifications or intentionally left out potentially.
These undefined system specifications are called \textit{holes},~and they can,~e.g.,~reflect an~unspecified component for wireless specification or a~partially implemented controller.
The~synthesis's primary purpose is through analysis reveals a~concrete subsystem with fully-defined behaviour and eventually reveals optimal designs when they are requested.
A~vital aspect of the design cycle is design space explorations,~i.e. exploring all possible designs.


A~\textit{one-by-one} approach can naively solve the~synthesis problem by analysing all unique designs~\cite{spl3,onebyone}.
On the contrary,~the whole design space can also be modelled as a~single \textit{all-in-one} Markov decision process~\cite{spl3,allinone}.
However,~enumerating all members of design space (realisations) is unfeasible due to its combinatorial explosion, and the size of such all-in-one MDP is proportional to the number of candidate designs.
Unfortunately,~the~double state-space explosion problem renders both of these approaches infeasible for large families.
Other approaches consider evolutionary search algorithms for the~synthesis of software systems~\cite{spl2}.
These methods remain incomplete and cannot efficiently solve more challenging systems,~e.g.,~which design satisfies the specification \textit{optimally}.

In this work,~we will focus on a~complete state-of-the-art approach for the~synthesis of probabilistic programs.
This approach was first introduced in \textit{Andriushchenko} master’s thesis~\cite{roman-DP}, followed~by its improvement in~\cite{tacas21}.
It combines two sophisticated methods providing an~analysis of whole design sub-families at once.
The~first method analyses each design from a~given sub-family individually and constructs critical sub-systems of counter-examples to prune all designs behaving incorrectly \,--\, the~so-called \textit{counter-example guided inductive synthesis} (CEGIS)~\cite{cegis}.
The~second method,~called abstraction refinement (AR)~\cite{cegar},~immediately analyses the~entire design space and refines it into design sub-families when the~analysis yields inconclusive results.
Both of these methods have shown convincing results,~although each faces certain limitations.
They are incomparable because one method can be more suitable for specific probabilistic programs classes and conversely.
The~approach presented in this work integrates both these methods and it manages to significantly outperform them,~sometimes by a~margin of orders of magnitude.


\chapter{Synthesis of Probabilistic Programs}


\chapter{Advanced Methods for Probabilistic Synthesis}


\chapter{Combined Probabilistic Synthesis}


\chapter{Tool Architecture}


\chapter{Experimental Evaluation}


\chapter{Final Considerations}


\section{Future Research}
\section{Conclusions}
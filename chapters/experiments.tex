\chapter{Experimental Evaluation}\label{chap:experiments}
This chapter will investigate the~performance of the~\emph{hybrid} approach for integrated synthesis methods extensions \,--\, multi-property and optimal synthesis \,--\, for various case studies and properties.
Moreover,~we will also introduce a~preliminary results of the~performance of the~designed approach for both \emph{parameter} and \emph{combined} synthesis.
Further on,~we demonstrate the~applicability of \toolname{} and interpret the~synthesis results for three of these case studies.
All experiments are run on a~Debian~GNU/Linux~10 machine with Intel(R) Core(TM) i7-3770K (8 cores at 3.50GHz) and using up to 32 GB RAM,~with all the~algorithms being executed single-threaded.

\section{Performance Evaluation of Advanced Methods}
This section aims to demonstrate the~performance of advanced integrated methods for both \textit{multi-property} and \textit{optimal} synthesis on various synthesis problems from different application domains.
In particular,~we compare its performance with the~\emph{one-by-one} enumeration representing the~baseline algorithm implemented in the~existing synthesis tools such as QFLan~\cite{qflan} and ProFeat~\cite{profeat}.
Experiments in previous papers~\cite{cegar,cegis} have shown that the~synthesis methods implemented in these tools have evident deficits on the~investigated benchmarks.
Moreover, experiments in~\cite{roman-DP} have shown that presented hybrid method significantly outperform both state-of-the art synthesis approaches \,--\, CEGIS and AR.
These facts is supported by comparing the~hybrid approach only with the~one-by-one enumeration we present in this section.
For each case study,~we report the~results for \textit{unfeasibility} problems with one property and two properties,~and \textit{optimal} synthesis problem and its \textit{relative} variant with $\varepsilon = 0.05$.
In all cases,~the~synthesis methods have to explore the~whole design space.
The~metrics marks with * represent the~qualified estimates.
We consider the following case studies:

\paragraph{Herman.}
This model represents distributed asynchronous protocol for rings with self-stabilisation~\cite{herman1,herman2}.
The~protocol is extended with a~choice for flipping several unfair coins,~and each station in the~ring includes one-bit memory.
All stations in the~ring have equivalent behaviour,~i.e.,~they are anonymous,~but they decide based on the~own local events and value of the~memory.
The~family maintains this anonymity and describes the~different variations of coin choice and updates of memory.
We are interested mainly in the~expected time until \emph{stabilisation}.
    
\begin{table}[h!]
\centering
\begin{tabular}{l|cccc}
    \hline \hline 
    & \multicolumn{1}{l}{\textbf{1 property}} & \multicolumn{1}{l}{\textbf{2 properties}} & \multicolumn{1}{l}{\textbf{optimal}} & \multicolumn{1}{l}{\textbf{$\mathbf{5\%}$-optimal}} \\ \hline
    \textbf{1-by-1} & 32h & 40h & 32h & \,--\, \\
    \textbf{Hybrid} & 90s & 105s & 21m & 8m \\ \hline \hline
\end{tabular}
\caption{\textbf{Herman}: Family size: $3.1M$, Number of parameters: $7$, Average MC size: $1.1k$.}
\end{table}

\paragraph{DPM.}
It represents a~partial information scheduler for a~disk power manager motivated by~\cite{dpm1}.
This model have been precise described in Section~\ref{sec:dpm}.
We are interested primarily in the~expected energy consumption and expected number of lost requests.

\begin{table}[h!]
\centering
\begin{tabular}{l|cccc}
    \hline \hline 
    & \multicolumn{1}{l}{\textbf{1 property}} & \multicolumn{1}{l}{\textbf{2 properties}} & \multicolumn{1}{l}{\textbf{optimal}} & \multicolumn{1}{l}{\textbf{$\mathbf{5\%}$-optimal}} \\ \hline
    \textbf{1-by-1} & 31d & 35d & 31d & \,--\, \\
    \textbf{Hybrid} & 72m & 84m & 9.7h & 6.2h \\ \hline \hline
\end{tabular}
\caption{\textbf{DPM}:  Family size: $43M$, Number of parameters: $16$, Average MC size: $3.6k$.}
\end{table}

\paragraph{Maze.}
It represents a planning problem typically modelled as POMDP~\cite{maze}.
The~family includes all deterministic strategies which are based on observation,~and containing small memory with a~fixed upper bound.
We are interested primarily in the~expected time to the \emph{goal}.

\begin{table}[h!]
\centering
\begin{tabular}{l|cccc}
    \hline \hline 
    & \multicolumn{1}{l}{\textbf{1 property}} & \multicolumn{1}{l}{\textbf{2 properties}} & \multicolumn{1}{l}{\textbf{optimal}} & \multicolumn{1}{l}{\textbf{$\mathbf{5\%}$-optimal}} \\ \hline
    \textbf{1-by-1} & 25h & 31h & 25h & \,--\, \\
    \textbf{Hybrid} & 63m & 65m & 78m & 59m \\ \hline \hline
\end{tabular}
\caption{\textbf{Maze}:  Family size: $9.4M$, Number of parameters: $22$, Average MC size: $0.2k$.}
\end{table}

\paragraph{Pole.}
It models balancing a pole in an unknown and noisy environment~\cite{pole}.
A~model controller optimises an~expected behaviour during the deployment without dependence on the~current (hidden) environment since it is preferred before the~finite set of environment behaviours.
The~family described schedulers that are independent of hidden information.
We are interested mainly in the~expected time until \emph{failure}.

\begin{table}[h!]
\centering
\begin{tabular}{l|cccc}
    \hline \hline 
    & \multicolumn{1}{l}{\textbf{1 property}} & \multicolumn{1}{l}{\textbf{2 properties}} & \multicolumn{1}{l}{\textbf{optimal}} & \multicolumn{1}{l}{\textbf{$\mathbf{5\%}$-optimal}} \\ \hline
    \textbf{1-by-1} & 23h & 30h & 23h & \,--\, \\
    \textbf{Hybrid} & 7s & 8s & 11m & 60s \\ \hline \hline
\end{tabular}
\caption{\textbf{Pole}:  Family size: $1.3M$, Number of parameters: $17$, Average MC size: $5.6k$.}
\end{table}

\paragraph{Grid.}
It represents a~model describing again partially observable MDPs (POMDPs)~\cite{pomdp1}.
There is an~agent who tries to locate a~target cell in a~4x4 grid.
We are interested in lower- and upper- bounded properties on the~expected number of steps.

\begin{table}[h!]
\centering
\begin{tabular}{l|cccc}
    \hline \hline 
    & \multicolumn{1}{l}{\textbf{1 property}} & \multicolumn{1}{l}{\textbf{2 properties}} & \multicolumn{1}{l}{\textbf{optimal}} & \multicolumn{1}{l}{\textbf{$\mathbf{5\%}$-optimal}} \\ \hline
    \textbf{1-by-1} & 16m & 19m & 16m & \,--\, \\
    \textbf{Hybrid} & 31s & 35s & 47s & 9s \\ \hline \hline
\end{tabular}
\caption{\textbf{Grid}: Family size: $65k$, Number of parameters: $8$, Average MC size: $1.2k$.}
\end{table}
    

\paragraph{Evaluation.}
As we have already said,~we are based on what has already been shown in previous articles~\cite{roman-DP,cegar,cegis}.
As a~basis,~we know that a~hybrid oracle is orders of magnitude faster than one-by-one enumeration when analysing single property.
Based on the~results obtained in the~framework of all considered case studies,~we can draw the~following conclusions,~which confirm the~tables above.
An~integrated \emph{multi-property} synthesis slows down both approaches,~although the~hybrid slowdown is almost negligible.
An~\emph{optimal} synthesis slows down only the~hybrid approach,~yet it is still incomparably faster than naive one-by-one enumeration.
Moreover,~a~hybrid oracle supports $\varepsilon$-optimal synthesis,~which is even faster.

\section{Performance Evaluation of Combined Synthesis}

\section{Applicability}


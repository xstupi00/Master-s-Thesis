%==============================================================================
% tento soubor pouzijte jako zaklad
% this file should be used as a base for the thesis
% Autoři / Authors: 2008 Michal Bidlo, 2019 Jaroslav Dytrych
% Kontakt pro dotazy a připomínky: sablona@fit.vutbr.cz
% Contact for questions and comments: sablona@fit.vutbr.cz
%==============================================================================
% kodovani: UTF-8 (zmena prikazem iconv, recode nebo cstocs)
% encoding: UTF-8 (you can change it by command iconv, recode or cstocs)
%------------------------------------------------------------------------------
% zpracování / processing: make, make pdf, make clean
%==============================================================================
% Soubory, které je nutné upravit nebo smazat: / Files which have to be edited or deleted:
%   projekt-20-literatura-bibliography.bib - literatura / bibliography
%   projekt-01-kapitoly-chapters.tex - obsah práce / the thesis content
%   projekt-01-kapitoly-chapters-en.tex - obsah práce v angličtině / the thesis content in English
%   projekt-30-prilohy-appendices.tex - přílohy / appendices
%   projekt-30-prilohy-appendices-en.tex - přílohy v angličtině / appendices in English
%==============================================================================
% \documentclass[]{fitthesis} % bez zadání - pro začátek práce, aby nebyl problém s překladem
% \documentclass[english]{fitthesis} % without assignment - for the work start to avoid compilation problem
%\documentclass[zadani]{fitthesis} % odevzdani do wisu a/nebo tisk s barevnými odkazy - odkazy jsou barevné
% \documentclass[english,enslovak,zadani]{fitthesis} % for submission to the IS FIT and/or print with color links - links are color
%\documentclass[zadani,print]{fitthesis} % pro černobílý tisk - odkazy jsou černé
%\documentclass[english,zadani,print]{fitthesis} % for the black and white print - links are black
%\documentclass[zadani,cprint]{fitthesis} % pro barevný tisk - odkazy jsou černé, znak VUT barevný
%\documentclass[english,zadani,cprint]{fitthesis} % for the print - links are black, logo is color
% * Je-li práce psaná v anglickém jazyce, je zapotřebí u třídy použít 
%   parametr english následovně:
%   If thesis is written in English, it is necessary to use 
%   parameter english as follows:
%      \documentclass[english]{fitthesis}
% * Je-li práce psaná ve slovenském jazyce, je zapotřebí u třídy použít 
%   parametr slovak následovně:
%   If the work is written in the Slovak language, it is necessary 
%   to use parameter slovak as follows:
%      \documentclass[slovak]{fitthesis}
% * Je-li práce psaná v anglickém jazyce se slovenským abstraktem apod., 
%   je zapotřebí u třídy použít parametry english a enslovak následovně:
%   If the work is written in English with the Slovak abstract, etc., 
%   it is necessary to use parameters english and enslovak as follows:
\documentclass[english,enslovak,zadani]{fitthesis}

% Základní balíčky jsou dole v souboru šablony fitthesis.cls
% Basic packages are at the bottom of template file fitthesis.cls
% zde můžeme vložit vlastní balíčky / you can place own packages here


%PAYNT (Probabilistic progrAm sYNThesizer)
\newcommand{\toolname}{\textsc{PAYNT}}
\newcommand{\storm}{\textsc{Storm}}
\newcommand{\prism}{\textsc{Prism}}
\newcommand{\jani}{\textsc{Jani}}
\newcommand{\rodes}{\textsc{Rodes}}

\newcommand*\circled[1]{\tikz[baseline=(char.base)]{
            \node[shape=circle,draw,inner sep=2pt] (char) {#1};}}

% numbers
\newcommand{\nat}{\mathbb{N}_0}
\newcommand{\rat}{\mathbb{Q}}
\newcommand{\real}{\mathbb{R}}
\newcommand{\realpos}{\real_{\geq 0}}
\newcommand{\unitinterval}{[0,1]}
\newcommand{\floor}[1]{\lfloor #1 \rfloor}

% vectors
\newcommand{\vect}[1]{\boldsymbol{#1}}
\newcommand{\norm}[1]{\left\| #1 \right\|_1}

% probability
\newcommand{\prob}{\mathbb{P}}
\newcommand{\rew}{\mathbb{E}}
\newcommand{\supp}{\mathrm{supp}}

% properties
\renewcommand{\phi}{\varphi}
% \newcommand{\F}[1]{\Diamond#1}
\newcommand{\F}[1]{\mathrm{F}\ #1}
\newcommand{\reachability}[3]{\prob_{#1 #2}[\F{#3}]}
\newcommand{\reward}[3]{\rew_{#1 #2}[\F{#3}]}
\newcommand{\safety}[2]{\reachability{\leq}{#1}{#2}}
\newcommand{\liveness}[2]{\reachability{\geq}{#1}{#2}}
\newcommand{\safetys}{\safety{\lambda}{T}}
\newcommand{\livenesss}{\liveness{\lambda}{T}}
\DeclareMathOperator*{\argmax}{arg\,max}
\DeclareMathOperator*{\argmin}{arg\,min}
\newcommand{\sketch}{\mathrm{P}}

% transition systems
\newcommand{\sins}{s \in S}
\newcommand{\sinit}{s_{0}}
\newcommand{\sBot}{s_{\bot}}
\newcommand{\sTop}{s_{\top}}
\newcommand{\statesum}[1]{\sum_{#1 \in S}}
\newcommand{\lastpi}{\mathrm{last}(\pi)}
\newcommand{\paths}[1]{\mathrm{Paths}^{#1}}
\newcommand{\pathsfin}[1]{\mathrm{Paths}^{#1}_{\mathrm{fin}}}

% mc
\renewcommand{\pm}{\vect{P}}
\newcommand{\mc}{(S,\sinit,\pm)}
\newcommand{\subchain}[2]{#1 {\downarrow} #2}

% mdp
\newcommand{\mpm}{\mathcal{P}}
\newcommand{\mdp}{(S,\sinit,Act,\mpm)}
\newcommand{\sigmamin}{\sigma_{\min}}
\newcommand{\sigmamax}{\sigma_{\max}}

% other
\newcommand{\vz}{\vect{0}}
\newcommand{\vo}{\vect{1}}

\newcommand{\vx}{\vect{x}}
\newcommand{\vxmin}{\vect{x}_{\min}}
\newcommand{\vxmax}{\vect{x}_{\max}}
\newcommand{\vxi}{\vx^{(i)}}
\newcommand{\vg}{\vect{\gamma}}
\newcommand{\vc}{\vect{\chi}}
\newcommand{\vu}{\vect{\upsilon}}

% family
\newcommand{\fml}{\mathcal{F}}
\newcommand{\fpm}{\mathcal{B}}
\newcommand{\family}{(S,\sinit,K, V, \fpm)}

% realization
\newcommand{\rlz}{\mathcal{R}}
\newcommand{\rlzf}{\overline{\rlz}}
\newcommand{\fmlr}{\fml_r}

% CEGIS, CEGAR
\newcommand{\generalize}[2]{#1 {\uparrow} #2}
\newcommand{\generalizes}{\generalize{r}{\overline{K}}}
\newcommand{\quotientmdp}{(S,\sinit,\rlzf,\mpm)}

% algorithms
\newcommand{\successRate}[1]{\sigma_{#1}}

% Integrated: parameters
\newcommand{\mml}{mdpMClimit}
\newcommand{\api}{activatePerIter}

\usepackage{amsthm}

\theoremstyle{definition}
\newtheorem{definition}{Definition}
\newtheorem{theorem}{Theorem}
\newtheorem{example}{Example}

\makeatletter
\newenvironment{chapquote}[2][2em]
  {\setlength{\@tempdima}{#1}%
   \def\chapquote@author{#2}%
   \parshape 1 \@tempdima \dimexpr\textwidth-2\@tempdima\relax%
   \itshape}
  {\par\normalfont\hfill--\ \chapquote@author\hspace*{\@tempdima}\par\bigskip}
\makeatother

% Kompilace po částech (rychlejší, ale v náhledu nemusí být vše aktuální)
% Compilation piecewise (faster, but not all parts in preview will be up-to-date)
% \usepackage{subfiles}

% Nastavení cesty k obrázkům
% Setting of a path to the pictures
%\graphicspath{{obrazky-figures/}{./obrazky-figures/}}
%\graphicspath{{obrazky-figures/}{../obrazky-figures/}}

%---rm---------------
\renewcommand{\rmdefault}{lmr}%zavede Latin Modern Roman jako rm / set Latin Modern Roman as rm
%---sf---------------
\renewcommand{\sfdefault}{qhv}%zavede TeX Gyre Heros jako sf
%---tt------------
\renewcommand{\ttdefault}{lmtt}% zavede Latin Modern tt jako tt

% vypne funkci šablony, která automaticky nahrazuje uvozovky,
% aby nebyly prováděny nevhodné náhrady v popisech API apod.
% disables function of the template which replaces quotation marks
% to avoid unnecessary replacements in the API descriptions etc.
\csdoublequotesoff

\usepackage{algorithm,algorithmic}
\usepackage{tikz}
\usetikzlibrary{automata,positioning,shapes,fit,calc}
\usepackage{url}

\newcommand{\MC}[1]{\textcolor{red}{MC: #1}}


\renewcommand{\algorithmiccomment}[1]{\textbf{\hfill // #1}}% Updated definition
\usepackage[labelfont=bf]{caption}

% =======================================================================
% balíček "hyperref" vytváří klikací odkazy v pdf, pokud tedy použijeme pdflatex
% problém je, že balíček hyperref musí být uveden jako poslední, takže nemůže
% být v šabloně
% "hyperref" package create clickable links in pdf if you are using pdflatex.
% Problem is that this package have to be introduced as the last one so it 
% can not be placed in the template file.
\ifWis
\ifx\pdfoutput\undefined % nejedeme pod pdflatexem / we are not using pdflatex
\else
  \usepackage{color}
  \usepackage[unicode,colorlinks,hyperindex,plainpages=false,pdftex]{hyperref}
  \definecolor{hrcolor-ref}{RGB}{223,52,30}
  \definecolor{hrcolor-cite}{HTML}{2F8F00}
  \definecolor{hrcolor-urls}{HTML}{092EAB}
  \hypersetup{
	linkcolor=hrcolor-ref,
	citecolor=hrcolor-cite,
	filecolor=magenta,
	urlcolor=hrcolor-urls
  }
  \def\pdfBorderAttrs{/Border [0 0 0] }  % bez okrajů kolem odkazů / without margins around links
  \pdfcompresslevel=9
\fi
\else % pro tisk budou odkazy, na které se dá klikat, černé / for the print clickable links will be black
\ifx\pdfoutput\undefined % nejedeme pod pdflatexem / we are not using pdflatex
\else
  \usepackage{color}
  \usepackage[unicode,colorlinks,hyperindex,plainpages=false,pdftex,urlcolor=black,linkcolor=black,citecolor=black]{hyperref}
  \definecolor{links}{rgb}{0,0,0}
  \definecolor{anchors}{rgb}{0,0,0}
  \def\AnchorColor{anchors}
  \def\LinkColor{links}
  \def\pdfBorderAttrs{/Border [0 0 0] } % bez okrajů kolem odkazů / without margins around links
  \pdfcompresslevel=9
\fi
\fi
% Řešení problému, kdy klikací odkazy na obrázky vedou za obrázek
% This solves the problems with links which leads after the picture
\usepackage[all]{hypcap}

% Informace o práci/projektu / Information about the thesis
%---------------------------------------------------------------------------
\projectinfo{
  %Prace / Thesis
  project={DP},            %typ práce BP/SP/DP/DR  / thesis type (SP = term project)
  year={2021},             % rok odevzdání / year of submission
  date=\today,             % datum odevzdání / submission date
  %Nazev prace / thesis title
  title.cs={Pokročilé metódy pre syntézu pravdepodobnostných programov},  % název práce v češtině či slovenštině (dle zadání) / thesis title in czech language (according to assignment)
  title.en={Advanced Methods for Synthesis of Probabilistic Programs}, % název práce v angličtině / thesis title in english
  %title.length={14.5cm}, % nastavení délky bloku s titulkem pro úpravu zalomení řádku (lze definovat zde nebo níže) / setting the length of a block with a thesis title for adjusting a line break (can be defined here or below)
  %sectitle.length={14.5cm}, % nastavení délky bloku s druhým titulkem pro úpravu zalomení řádku (lze definovat zde nebo níže) / setting the length of a block with a second thesis title for adjusting a line break (can be defined here or below)
  %Autor / Author
  author.name={Šimon},   % jméno autora / author name
  author.surname={Stupinský},   % příjmení autora / author surname 
  author.title.p={Bc.}, % titul před jménem (nepovinné) / title before the name (optional)
  %author.title.a={Ph.D.}, % titul za jménem (nepovinné) / title after the name (optional)
  %Ustav / Department
  department={UITS}, % doplňte příslušnou zkratku dle ústavu na zadání: UPSY/UIFS/UITS/UPGM / fill in appropriate abbreviation of the department according to assignment: UPSY/UIFS/UITS/UPGM
  % Školitel / supervisor
  supervisor.name={Milan},   % jméno školitele / supervisor name 
  supervisor.surname={Češka},   % příjmení školitele / supervisor surname
  supervisor.title.p={RNDr.},   %titul před jménem (nepovinné) / title before the name (optional)
  supervisor.title.a={Ph.D.},    %titul za jménem (nepovinné) / title after the name (optional)
  % Klíčová slova / keywords
  keywords.cs={automatizovaná syntéza, pravdepododnostné programy, Markovské modely, model checking}, % klíčová slova v českém či slovenském jazyce / keywords in czech or slovak language
  keywords.en={automated synthesis, probabilistic programs, Markov models, model checking}, % klíčová slova v anglickém jazyce / keywords in english
  %keywords.en={Here, individual keywords separated by commas will be written in English.},
  % Abstrakt / Abstract
  abstract.cs={
Pravdepodobnostné programy zohrávajú rozhodujúcu úlohu v rôznych technických doménach, ako napríklad počítačové siete, vstavané systémy, stratégie riadenia spotreby energie alebo softvérové produčkné linky.
\toolname{} je nástroj na automatizovanú syntézu pravdepodobnostných programov vyhovujúcich zadaným špecifikáciam.
V tejto práci rozširujeme tento nástroj predovšetkým o podporu optimálnej syntézy a syntézy viacerých špecifikácií.
Ďalej sme navrhli a implementovali novú metódu, ktorá dokáže efektívne syntetizovať  parametre so spojitým definičným oborom ovplyvňujúce pravdepodobnostné prechody popri syntéze topológie programov, t.j., podporu pre syntézu topológie aj parametrov súčasne.
Demonštrujeme užitočnosť a výkonnosť nástroja PAYNT na širokej škále prípadových štúdií z rôznych aplikačných domén ktoré majú uplatnenie v reálnom svete.
Pri náročných problémoch syntézy môže PAYNT výrazne znížiť dobu behu až z dní na minúty a zároveň zaistiť úplnosť procesu syntézy.
  
  }, % abstrakt v českém či slovenském jazyce / abstract in czech or slovak language
  abstract.en={
Probabilistic programs play a~crucial role in various engineering domains,~including computer networks,~embedded systems,~power management policies,~or software product lines.
\toolname{} is a tool for the~automatic synthesis of probabilistic programs satisfying the given specifications. 
In this thesis,~we extend this tool primarily to support optimal synthesis and synthesis for multi-property specifications.
Further, we have proposed and implemented a~new method that can efficiently synthesise continuous parameters affecting the transition probabilities alongside the synthesis of a program topology,~i.e.,~support of both topology and parameter synthesis at the same time.
We demonstrate the~usefulness and performance of \toolname{} on a~wide range of real-world case studies from various application domains.
For challenging synthesis problems,~\toolname{} can significantly decrease the~run-time from days to minutes while ensuring the~completeness of the~synthesis process.
  
  }, % abstrakt v anglickém jazyce / abstract in english
  %abstract.en={An abstract of the work in English will be written in this paragraph.},
  % Prohlášení (u anglicky psané práce anglicky, u slovensky psané práce slovensky) / Declaration (for thesis in english should be in english)
  declaration={Hereby I declare that this master's thesis was prepared as an~original author’s work under the~supervision of RNDr. Milan Češka, Ph.D.
  The~supplementary information was provided by consultant Ing. Roman Andriuschenko.
  All the~relevant information sources, which were used during preparation of this thesis, are properly cited and included in the~list of references.},
  %declaration={I hereby declare that this Bachelor's thesis was prepared as an original work by the author under the supervision of Mr. X
% The supplementary information was provided by Mr. Y
% I have listed all the literary sources, publications and other sources, which were used during the preparation of this thesis.},
  % Poděkování (nepovinné, nejlépe v jazyce práce) / Acknowledgement (optional, ideally in the language of the thesis)
  acknowledgment={I would like to express my sincere gratitude to my supervisor,~RNDr. Milan Češka, Ph.D.,~for his guidance and continuous support throughout my master’s studies,~and Ing. Roman Andriuschenko for his advices and cooperation on the~development of our work. Last but not least, a sincere thanks goes to my girlfriend Dominika, family and friends for their never-ending support and encouragement during the inevitable hard times.},
  %acknowledgment={Here it is possible to express thanks to the supervisor and to the people which provided professional help
%(external submitter, consultant, etc.).},
  % Rozšířený abstrakt (cca 3 normostrany) - lze definovat zde nebo níže / Extended abstract (approximately 3 standard pages) - can be defined here or below
  %extendedabstract={Do tohoto odstavce bude zapsán rozšířený výtah (abstrakt) práce v českém (slovenském) jazyce.},
  %faculty={FIT}, % FIT/FEKT/FSI/FA/FCH/FP/FAST/FAVU/USI/DEF
  faculty.cs={Fakulta informačních technologií}, % Fakulta v češtině - pro využití této položky výše zvolte fakultu DEF / Faculty in Czech - for use of this entry select DEF above
  faculty.en={Faculty of Information Technology}, % Fakulta v angličtině - pro využití této položky výše zvolte fakultu DEF / Faculty in English - for use of this entry select DEF above
%   department.cs={Ústav matematiky}, % Ústav v češtině - pro využití této položky výše zvolte ústav DEF nebo jej zakomentujte / Department in Czech - for use of this entry select DEF above or comment it out
%   department.en={Institute of Mathematics} % Ústav v angličtině - pro využití této položky výše zvolte ústav DEF nebo jej zakomentujte / Department in English - for use of this entry select DEF above or comment it out
}

% Rozšířený abstrakt (cca 3 normostrany) - lze definovat zde nebo výše / Extended abstract (approximately 3 standard pages) - can be defined here or above
\extendedabstract{Náhodnosť je kľúčová pre výskumné oblasti, ako sú napríklad pravdepodobnostné programovanie, spoľahlivosť (systémové komponenty s neistotou), distribuované výpočty (porušenie symetrie), a plánovanie (neznáme a neisté prostredia). 
Pravdepodobnostné programy sú výkonným modelovacím nástrojom pre systémy obsahujúce pravdepodobnostnú neistotu. 
Systémy s nespoľahlivým a nepredvídateľným správaním si vyžadujú použitie takéhoto matematického aparátu založeného na teórii pravdepodobnosti. 
Návrh takýchto systémov vykazujúcich žiadúce správanie \,--\, napr. výber optimálnej stratégie riadenia spotreby energie alebo výber sieťového protokolu zvyšujúceho priepustnosť paketov \,--\, je náročný z hľadiska uvažovania o viacerých alternatívnych prevedeniach daného systému. Aplikácie takýchto systémov pokrývajú širokú škálu výskumných oblastí, vrátane analýzy (kvantitatívnych) softvérových produktových liniek~\cite{sw-product-lines,spl2}, syntézy stratégie pri plánovaní za čiastočnej pozorovateľnosti~\cite{pomdp1,pomdp2} alebo návrh komunikačných protokolov~\cite{herman1,herman2}.

Sada deklaratívnych časových obmedzení často vyjadruje účinnosť a správnosť pravdepodobnostných programov.
Nástroje pre kontrolu modelov pravdepodobnostných systémov, ako \storm{}~\cite{STORM} alebo \prism{}~\cite{KNP11}, vykonávajú automatizované overenie takýchto obmedzení.
Tieto nástroje pre kontrolu pravdepodobnostných modelov však vyžadujú model alebo systém s pevnou topológiou, na rozdiel od požiadaviek na použitie týchto nástrojov pri modelovaní pravdepodobnostných programov.
Vývojári by mali čo najskôr počas procesu vývoja systému overiť jeho návrhy, aby náklady na vývoj boli čím lacnejšie a udržateľnejšie. Návrh systému je v počiatočných fázach vývoja prevažne neúplný, pretože vo väčšine prípadov nie sú známe všetky špecifikácie systému alebo sú zámerne vynechané. 
Tieto nedefinované špecifikácie systému sa nazývajú diery a môžu napríklad odrážať nešpecifikovanú komponentu alebo čiastočne implementovaný radič. 
Primárnym účelom syntézy je odhaliť konkrétny (optimálny) subsystém s plne definovaným správaním. Zásadným aspektom návrhového cyklu je prieskum množiny návrhov, t.j. preskúmanie všetkých možných návrhov uvažovaného systému.
Keď považujeme Markovský reťazec za matematický aparát pravdepodobnostného programu, potom návrhový priestor predstavuje rodinu takýchto reťazcov a úlohou syntézy je nájsť ten reťazec, ktorý spĺňa dané špecifikácie.

Typicky sa syntéza programu považuje za problém pri deduktívnom dokazovaní, ktoré odvodzuje program, ktorý vyhovuje zadaným špecifikáciam, od konštruktívneho dôkazu~\cite{deductive-synthesis}.
V poslednej dobe prichádza do popredia alternatívny prístup k syntéze, keď návrhári, okrem špecifikácie, poskytujú aj syntaktickú šablónu pre požadovaný program.
Uvažujeme o prístupe využívajúcom náčrt systému, keď návrhár skonštruuje čiastkový program s neúplnými podrobnosťami a syntetizátor doplní chýbajúce podrobnosti oproti uvedeným špecifikáciám~\cite{sketching1}.
Ďalej môžeme problémy syntézy pre parametrické pravdepodobnostné systémy rozdeliť do dvoch kategórií. 
Najskôr zvážime úlohu syntézy topológie za predpokladu konečnej sady parametrov, ktoré ovplyvňujú topológiu modelu, kde jednotlivé parametre predstavujú nedefinované špecifikácie systému.
Táto úloha sa zameriava na rodiny Markovských reťazcov, ktoré majú rozdielne topológie stavového priestoru a v dôsledku toho rôzne množiny dosiahnuteľných stavov.
Avšak iná oblasť úloh syntézy uvažuje o Markovskom reťazci s pevnou topológiou, ale nedefinovanými pravdepodobnosťami prechodu v rámci modelu.
}

% nastavení délky bloku s titulkem pro úpravu zalomení řádku - lze definovat zde nebo výše / setting the length of a block with a thesis title for adjusting a line break - can be defined here or above
%\titlelength{14.5cm}
% nastavení délky bloku s druhým titulkem pro úpravu zalomení řádku - lze definovat zde nebo výše / setting the length of a block with a second thesis title for adjusting a line break - can be defined here or above
%\sectitlelength{14.5cm}

% řeší první/poslední řádek odstavce na předchozí/následující stránce
% solves first/last row of the paragraph on the previous/next page
\clubpenalty=10000
\widowpenalty=10000

% checklist
\newlist{checklist}{itemize}{1}
\setlist[checklist]{label=$\square$}

\begin{document}
  % Vysazeni titulnich stran / Typesetting of the title pages
  % ----------------------------------------------
  \maketitle
  % Obsah
  % ----------------------------------------------
  \setlength{\parskip}{0pt}

  {\hypersetup{hidelinks}\tableofcontents}
  
  % Seznam obrazku a tabulek (pokud prace obsahuje velke mnozstvi obrazku, tak se to hodi)
  % List of figures and list of tables (if the thesis contains a lot of pictures, it is good)
  \ifczech
    \renewcommand\listfigurename{Seznam obrázků}
  \fi
  \ifslovak
    \renewcommand\listfigurename{Zoznam obrázkov}
  \fi
  % {\hypersetup{hidelinks}\listoffigures}
  
  \ifczech
    \renewcommand\listtablename{Seznam tabulek}
  \fi
  \ifslovak
    \renewcommand\listtablename{Zoznam tabuliek}
  \fi
  % {\hypersetup{hidelinks}\listoftables}

  \ifODSAZ
    \setlength{\parskip}{0.5\bigskipamount}
  \else
    \setlength{\parskip}{0pt}
  \fi

  % vynechani stranky v oboustrannem rezimu
  % Skip the page in the two-sided mode
  \iftwoside
    \cleardoublepage
  \fi

  % Text prace / Thesis text
  % ----------------------------------------------
  \ifenglish
    \chapter{Introduction}

Randomisation is essential to research areas such as \textit{probabilistic programming},~dependability (system components with uncertainty),~distributed computing (symmetry breaking),~and planning (unknown and noisy environments).
Probabilistic programs are powerful modelling apparatus of systems containing probabilistic uncertainty.
Systems with unreliable and unpredictable behaviour require the~use of such a~mathematical apparatus established on probability theory.
Designing such systems exhibiting a~desirable behaviour \,--\, e.g. selecting the~optimal power management strategy or a~network protocol increasing the~packet throughput \,--\, is challenging for reasoning over multiple alternative designs.
Their applications cover a~broad range of research areas,~including,~e.g. analysis of (quantitative) software product lines~\cite{spl1,spl2},~strategy synthesis in planning under partial observability~\cite{pomdp1,pomdp2},~or design of communication protocols~\cite{herman1,herman2}.

A~set of declarative temporal constraints often expresses the~efficiency and correctness of the~probabilistic programs.
The~model checkers for probabilistic systems,~such as \storm{}~\cite{STORM} or \prism{}~\cite{KNP11},~provide automated verification of such constraints.
However,~these probabilistic model checkers require a~fixed model or a~fixed program,~contrary to their usage requirements when modelling probabilistic programs.
Developers need to verify the~system designs as early as possible at the~developing process to maintain its costs tractable and as best as possible.
System designs are prevailingly incomplete at the~initial development phases because,~in most cases,~there are no known all system specifications or intentionally left out potentially.
These undefined system specifications are called \textit{holes},~and they can,~e.g.,~reflect an~unspecified component for wireless specification or a~partially implemented controller.
The~synthesis's primary purpose is through analysis reveals a~concrete subsystem with fully-defined behaviour and eventually reveals optimal designs when they are requested.
A~vital aspect of the design cycle is design space explorations,~i.e. exploring all possible designs.
When considering a~Markov chain (MC) as the~mathematical apparatus of a~probabilistic program,~then the~design space represents a~family of such chains,~and the~synthesis task is to find the~one that satisfies a~given specifications.

A~\textit{one-by-one} approach can naively solve the~synthesis problem by analysing all unique designs~\cite{spl3,onebyone}.
On the contrary,~the whole design space can also be modelled as a~single \textit{all-in-one} Markov decision process~\cite{spl3,allinone}.
However,~enumerating all members of design space (realisations) is unfeasible due to its combinatorial explosion, and the size of such all-in-one MDP is proportional to the number of candidate designs.
Unfortunately,~the~double state-space explosion problem renders both of these approaches infeasible for large families.
Other approaches consider evolutionary search algorithms for the~synthesis of software systems~\cite{spl2}.
These methods remain incomplete and cannot efficiently solve more challenging systems,~e.g.,~which design satisfies the specification \textit{optimally}.

In this work,~we will focus on a~complete state-of-the-art approach for the~synthesis of probabilistic programs.
This approach was first introduced in \textit{Andriushchenko} master’s thesis~\cite{roman-DP}, followed~by its improvement in~\cite{tacas21}.
It combines two sophisticated methods providing an~analysis of whole design sub-families at once.
The~first method analyses each design from a~given sub-family individually and constructs critical sub-systems of counter-examples to prune all designs behaving incorrectly \,--\, the~so-called \textit{counter-example guided inductive synthesis} (CEGIS)~\cite{cegis}.
The~second method,~called abstraction refinement (AR)~\cite{cegar},~immediately analyses the~entire design space and refines it into design sub-families when the~analysis yields inconclusive results.
Both of these methods have shown convincing results,~although each faces certain limitations.
They are incomparable because one method can be more suitable for specific probabilistic programs classes and conversely.
The~approach presented in this work integrates both these methods and it manages to significantly outperform them,~sometimes by a~margin of orders of magnitude.

All these presented methods consider \textit{topology synthesis} task assuming a~finite set of parameters that affect the~model topology,~where the~individual parameters represent the~undefined system specifications.
This task focuses on Markov chains families having different topologies of the~state space and,~as a~consequence,~different sets of reachable states.
However,~another area of synthesis tasks considers a~Markov chain with fixed topology but undefined transition probabilities.
This area has been discussed by approaches of model repairing~\cite{model-repair-1,pathak-et-al-nfm-2015} and techniques for \textit{parameter synthesis}~\cite{ceska2014robustness,Quatmann2016}.
When modelling real-world systems,~combining the two tasks can quickly occur,~but the~support to solve such a~\textit{combined synthesis} task does not exist.

\subsubsection*{Key Contributions.}
This thesis considers a~novel integrated method introduced in the~previous works~\cite{roman-DP,tacas21} as a~base stone.
Initially,~this method was designed only for feasibility synthesis task with one specification.
However,~probabilistic programs often have to satisfy specifications expressed as a~conjunction of several temporal logic constraints.
Therefore, we designed an~extension of this method to support \textit{multi-property specifications} and \textit{optimal synthesis} task.
The~designed extension for multi-properties is performed in the~same loop as the~origin single-property synthesis, with~necessary modifications: \textit{AR} need to analyse satisfiable specifications within inference sub-families no longer,~and \textit{CEGIS} analyses each specification individually and constructs counter-examples whenever a~given specification is unsatisfiable.
Consequently,~the~novel integrated method inherits the~benefits of \textit{AR} and \textit{CEGIS} in its favour also at multi-property synthesis.
\textit{Optimal synthesis} is a~particular instance of \textit{multi-property synthesis},~and it can find an application in various domains.
In particular,~specification set includes so-called violation property representing the currently optimal solution,~and its threshold is updated whenever a~new optimal solution is found.
Moreover,~we designed support for the~relaxed variant of the~optimal synthesis,~so-called $\varepsilon$-optimal synthesis,~which is in most cases even faster.
We evaluate designed extensions on an~extensive set of real-world case studies. 
We confirm the~results of a~novel approach presented in the~previous works and found the~following conclusions relating to these extensions.
A~novel integrated method is orders of magnitude faster than one-by-one enumeration when analysing the~single property.
A~multi-property synthesis slows down both approaches,~although the~novel method slow down is almost negligible.
An~optimal synthesis slows down only the~integrated approach,~yet it is still incomparably faster than enumeration.
The~assumption that $\varepsilon$-optimal synthesis can significantly speed up the~whole optimal synthesis process was also confirmed.

\todo{Parameter synthesis ...}


\subsubsection*{Structure of this paper.}
In Chapter~\ref{chap:synthesis},~we formulate a~probabilistic synthesis problem and introduce a~state-of-the-art novel integrated method based on two modern approaches \textit{CEGIS} and \textit{AR}.
Chapter~\ref{chap:advanced} develops vital ideas associated with integrating the \textit{multi-property} synthesis and \textit{optimal} synthesis within the presented integrated method.
Then, in Chapter~\ref{chap:combined}, we develop critical ideas associated with integrating the combined synthesis \,--\, consists of topology and parameter synthesis \,--\, within the considered method.
Chapter~\ref{chap:paynt} introduces a~new tool called \toolname{} and its architecture,~which implemented the~presented methods.
Chapter~\ref{chap:experiments} evaluates our solutions on a~broad range set of real-world case studies and compares them with the~baseline enumeration approach.
Finally,~Chapter~\ref{chap:conclusion} closes this thesis with the~notes
and issues that can serve as a~baseline point for the~follow-up research and potential improvement of designed solutions.

\chapter{Synthesis of Probabilistic Programs}\label{chap:synthesis}

\section{Problem Statement}
This section formalises necessary ingredients and the~problem statement for probabilistic synthesis.
We introduce definitions that assume parameters affecting MCs' topology,~and their adjustable versions for parameter synthesis we will introduce in Section~\ref{chap:combined}.
The following definitions are taken from the existing sources, mainly from~\cite{roman-DP,tacas21}, where a more detailed description can also be found.

\subsection{Markov Chains}

\begin{definition}[Distribution]
\cite{tacas21} 
    A \textit{discrete} distribution over a~finite or countably infinite set $X$ is a~function $\mu: X \rightarrow [0,1]$ s.t. $\sum_{x \in X} \mu(x) = \mu(X) = 1$.
    The~set of all distributions on $X$ is denoted $Distr(X)$.
    The support of a distribution $\mu$ is $supp(\mu) = \{ x \in X \lvert \mu(x) > 0 \}$.
    % A distribution is \textit{Dirac} if $\lvert supp(\mu) \rvert = 1$.
\end{definition}

\begin{example}
    Let $X = \{x_0, x_1, x_0\}$ be a finite set.
    Let function $\mu: X \rightarrow [0,1]$ defined as $\mu: [x_0 \mapsto \frac{1}{2}, x_1 \mapsto 0, x_2 \mapsto \frac{1}{2}]$ be a \textit{probability distribution} on $X$, i.e. $\mu \in Distr(X)$.
    The support of $\mu$ is $supp(\mu) = \{x_0, x_2 \}$, and for simplification, we writes such distributions as $\mu = \frac{1}{2} : x_0 + \frac{1}{2} : x_2$.
\end{example}


\begin{definition}[MC]
\cite{tacas21}
    A Markov chain (MC) D is a~triple $\mc$,~where $S$ is a~finite set of states,~$s_0 \in S$ is an~initial state,~and $\mathbf{P}: S \rightarrow Distr(S)$ is a~transition probability matrix.
    We write $\mathbf{P}(s, t)$ to denote $\mathbf{P}(s)(t)$.
    The state $s$ is \textit{absorbing} if $\mathbf{P}(s, s) = 1$.
\end{definition}

Instinctively,~we can imagine an~$MC$ as a~state-transition system with the~following semantics.
A~probability distribution for each state $\forall s \in S: \mathbf{P}(S)$ represents a~stochastic choice of firing the~transition from such state $s \in S$ to one of its successors states $s' \in supp(\mathbf{P(S)})$.
Consequently,~an~$MC$ defined with such semantics has a~\textit{Markov property} (memorylessness), which is essential when modelling systems and efficient analysis.
This property declares that the~probability of the~transition from state $s \in S$ to state $s' \in supp(\mathbf{P(S)})$ depends only on the~current state,~and it is independent of the~taken path of chain to state $s$.
We can see that each state of an $MC$ disposes of a unique probability distribution over its possible successor states.
In the~following definition,~we define an~extension of $MCs$ introducing a~non-deterministic choice between several probability distributions over each state.

\begin{definition}[MDP]
\cite{roman-DP}
    A Markov decision process (MDP) $M$ is a~quadruple $\mdp$ ~where $S$ is a~finite set of states,~$s_0 \in S$ is an~initial state,~$Act$ is a~finite set of actions,~and $\mathcal{P}: S \times Act  \nrightarrow Distr(S)$ represents a~(partial) transition probability function. 
\end{definition}

A~set $\mathit{Act(s) = \{ a \in Act \; \lvert \; \mathcal{P}(s, a) \neq \; \perp \}}$ for state $s \in S$ represents the~\textit{available} actions.
When holds $\lvert \mathit{Act(s)} \rvert = 1$ for each state $s \in S$,~such MDP straightforward induces an~MC.
A~(in)finite sequence $\pi = s_0 \overset{a_0}{\rightarrow} s_1 \overset{a_1}{\rightarrow} \dots$,~where $s_i \in S$,~$a_i \in Act(s_i)$,~and $\mathbb{P}(s_i, a_i)(s_{i+1}) \neq 0$ for all $i \in \mathbb{N}$ represents the~\textit{path} of an~MDP M.
For finite $\pi$, $last(\pi)$ denotes the last state of $\pi$,~and the~set of (in)finite paths of M we denotes as $Paths_{fin}^{M}\,(Paths^M)$.

When an~$MDP$ is currently in the~state $s \in S$,~it has a~non-deterministic choice of an action $a \in Act(s)$ leading to the~one possible probability distribution $P(s)(a)$ over its successors' states.
These actions cause the~non-deterministic behaviour of an~$MDP$.
Still,~this property can be suppressed by applying a~\textit{scheduler},~which selects one specific action in each state,~i.e.,~transforms an~$MDP$ to an~$MC$.

\begin{definition}[Scheduler]
\cite{cegar}
A scheduler for an MDP $M = \mdp$ is a function $\sigma: Paths_{fin}^{M} \rightarrow Act$ such that $\sigma(\pi) \in Act(last(\pi))$ for all $\pi \in Paths_{fin}^{M}$.
Scheduler $\sigma$ is memory-less if $last(\pi) = last(\pi') \Longrightarrow \sigma(\pi) = \sigma(\pi')$ for all $\pi, \pi' \in Paths_{fin}^{M}$.
The set of all schedulers of M is $\Sigma^M$.
\end{definition}

\begin{definition}[Induced MC] \label{def:incuded_mc}
\cite{cegar}
An MC induced by MDP M and $\sigma \in \Sigma^M$ is defined by $M_{\sigma} = ( Paths_{fin}^{M}, s_0, \mathbf{P}^{\sigma})$ where:
\begin{align*}
    \mathbf{P^{\sigma}}(\pi, \pi') = 
    \begin{cases}
        \mathcal{P}(last(\pi), \sigma(\pi))(s') \quad & if \; \pi' = \pi \overset{\sigma(\pi)}{\rightarrow} s' \\
         0 \quad & otherwise.
    \end{cases}
\end{align*}
\end{definition}

\subsection{Families of Markov Chains}
This thesis considers a~parametric transition probability function as an~explicit representation of an~MCs family.
Such explicit representation relieves the~presentation and provides to describe exciting and practical problems for probabilistic synthesis.
On the~other hand,~arbitrary probabilistic programs permit the~modelling of more complex and independent parameter structures~\cite{cegar}.
In this thesis and our implementation,~we consider a~more flexible high-level modelling language,~see Section~\todo{?}.

\begin{definition}[Family of MCs]
\cite{cegar}
    A~\emph{family of MCs} $\fml$ is a~quadruple $\family$  where $S$ is a~finite set of \emph{states}, $\sinit \in S$ is an~\emph{initial state},~$K$ is a~finite set of parameters with domains $T_k \subseteq S$ for each $k \in K$,~and $\fpm : S \rightarrow Distr(K)$ is a~family of transition probability functions.
\end{definition}

The~transition probability function $\fpm$ of MCs family $\fml$ maps each state $s \in S$ to the~probability distribution over parameters from $K$.
As we mentioned above,~these parameters represent undefined system specifications when the~probabilistic synthesis is considered.
This function $\fpm$ yields a~specific MC when we instantiate each parameter $k \in K$ with the~specific value from its domain $T_k \subseteq S$.
We call such instantiated MC as \textit{realisation},~and the~following definition describes it.

\begin{definition}[Realisation]
\cite{cegar}
A~\emph{realization} of a~family $\fml = \family$ of MCs is a~function $r: K \rightarrow S$ s.t.~$\forall k \in K :  r(k) \in T_k$. 
A~realization~$r$ yields a MC $\fmlr = (S,\sinit,\fpm(r))$,~where $\fpm(r)$ represents the~transition probability matrix where each parameter $k \in K$ is substituted with $r(k)$. 
Let $\rlzf$ denote the~set of all realisations for $\fml$.
\end{definition}

A~set $\prod_{k \in K} T_k$ representing all possible parameter combinations values has the~same semantics as a~set $\rlzf$ representing all family realisations.
In other words,~we can define the~\textit{size of the family} of MCs in the~following way: $\lvert \fml \rvert := \lvert \prod_{k \in K} T_k \rvert = \lvert \rlzf \rvert = \prod_{k \in K} \lvert T_k \rvert$.
The~family size $\lvert \fml \rvert$ is finite because of the~finiteness of each parameter domain $T_k \subseteq S$,~but it is exponential in the~number of parameters $\lvert K \rvert$.
We note $\rlz$ as the~sub-families induced from the~whole family $\rlzf$.
The~individual MCs within the~family share the~same state space $S$,~but their set of reachable states can vary.

\begin{example}[Family of MCs]\label{exam:mcfamily}
Let $\fml = \family$ be a~family of MCs,~where $S = \{s_0, s_1, s_2, s_3\}$ and $K = \{ k_0, k_1\}$ with domains $T_{k_0} = \{s_0, s_2\}$,~and $T_{k_1} = \{s_1, s_3\}$.
The parametric transition function $\fpm$ is defined as follows:
\begin{align*}
    \fpm(s_0) &= 0.5 : k_0 + 0.5 : k_1  &  \fpm(s_1)  &= 0.5 : k_1  + 0.5 : k_0 \\
    \fpm(s_2) &= 1.0: k_1   &  \fpm(s_3)  &= 1.0 : s_3
\end{align*}
Figure~\ref{fig:mcfamily} draws the~MCs family $\fml$ that correspond to all possible realisations: $\lvert T_{k_0} \rvert \cdot \lvert T_{k_1} \rvert = 2 \cdot 2 = 4$.
We note that these MCs have a~different topology of the~underlying state space,~resulting in different sets of reachable states.
\end{example}

\begin{figure}[ht!]
\centering
\includegraphics[width=0.9\textwidth]{figures/MCFamily.pdf}
\caption{A family $\fml$ of four various realisations. Unreachable states are greyed out.}%
\label{fig:mcfamily}%
\end{figure}

\textit{Quotient} MDP $M^\fml$ simulates the~behaviours of each member of the~family $\fml$ and can even pass between them during the~execution.
In more precise,~when the~path $s_0s_1s_2 \dots$ is executable in some family member $D_r$,~then it is also executable in $M^\fml$ as $s_0 \overset{r}{\rightarrow} s_1 \overset{r}{\rightarrow} \dots$.
However,~there may exist a~path $\pi$ that is executable in $M^\fml$,~but it is not realisable in neither of the~family members.
We can conclude that \textit{quotient} MDP over-approximates the~behaviours of the~members of family $\fml$.

\begin{definition}[Quotient MDP] \label{def:quotient_mdp}
\cite{roman-DP}
Let $\fml = \family$ be an~MCs family.
A~\textit{quotient} MDP $M^\fml$ of the~family $\fml$ is a~tuple $(S, s_{init}, \mathcal{R}^\fml, \mathcal{P})$,~where $\mathcal{P}(\cdot)(r) \equiv \mathcal{B}_r$.
\end{definition}

A~restriction of a~\textit{quotient} MDP concerning $\rlz \subseteq \rlz^\fml$ induces an~MDP which takes into account only transitions associated with $\rlz$.
We define the~usage of \textit{consistent} schedulers ensuring that execution of a~\textit{quotient} MDP always selects the~concrete realisation.
We note that a~\textit{consistent} scheduler yields a~specific family member for a~\textit{quotient} MDP,~which directly follows from Definition~\ref{def:quotient_mdp}.

\begin{definition}[Consistent scheduler]
\cite{roman-DP}
Let $\fml = \family$ be a~MCs family and let $M^\fml = (S, s_{init}, \mathcal{R}^\fml, \mathcal{P})$ be a quotient MDP of the~family $\fml$.
For $r \in \rlz^\fml$,~a (memory-less) scheduler $\sigma_r \in \sigma^{M^\fml}$ is called \textit{r-consistent} iff $\forall s \in S: \, \sigma(s) = r$.
A~scheduler is called \textit{consistent} iff it is consistent for some $r \in \rlz^\fml$.
\end{definition}

\subsection{Probabilistic Synthesis}
We use a~\textit{sketch}~\cite{sketching1,sygus}, ~which represents a~incomplete high-level description of a~probabilistic system,~to describe the~family of MCs.
It usually describes the~modelled system's fixed behaviour,~but it also contains the~undefined one represented by \textit{holes}.
They represent the~incomplete parts of a~program that must be instantiated so that the~final description satisfies a~given specification.
We note that the~holes correspond to parameters,~and their instantiation yields a~concrete Markov chain.
At follow, we formalise the sketch defining the set of designs:

\begin{definition}[Sketch]
Let $\sketch$ be a~sketch containing holes from the~set $\mathcal{H} = \left\{ H_k \right \}_k$ with $R_k$ being the~set of options available for hole $H_k$.
Let $\rlzf = \prod_k R_k$ denote the~set of all hole assignments (realizations),~$\sketch[r]$ denote the~program induced by a~substitution $r \in \rlzf$ and $\fmlr$ denote the underlying MC.
Note that the~set $\rlzf$ is exponential in $\lvert \mathcal{H} \rvert$.
\end{definition}

\todo{Some description ...}

\begin{definition}[Specification]
In our work,~we focus on the~conjunctions of specifications with \textit{reachability} and \textit{expected rewards}.
Let $T$ be a~set of target states,~then the~reachability property $\varphi \equiv \reachability{\bowtie}{\lambda}{T}$ where $\bowtie \in \{<, \leq, >, \geq\}$ and $0 \leq \lambda \leq 1$ expresses that the~probability of reaching $T$ refers to $\lambda \in [0,1]$ in agreement with $\bowtie$.
Expected reward property $\phi \equiv \reward {\bowtie}{\lambda}{T}$ expresses that the~expected reward accumulated before $T$ is reached refers to $\lambda \in \mathbb{R}^+$ in accordance with $\bowtie \, \in \{<, \leq \}$.
Let $\mathcal{P}[r]$ be a~program induced by the~realisation $r$,~we denote $\mathcal{P}[r] \models \varphi$ when this program satisfies the~specification $\varphi$.
Let $\varPhi = \{ \varphi_i \}_{i \in I}$ be a~finite set of specifications,~when $\forall i \in I: \mathcal{P}[r] \models \varphi_i$ then we write $\mathcal{P}[r] \models \varPhi$.
\end{definition}

We aim to two kinds of synthesis tasks for a~given probabilistic program described with a~realisation set $\rlzf$,~and a~specification set $\varPhi$.
The first task tries to identify just one realisation $r \in \rlzf$ that satisfy the~given specification set $\varPhi$.
This task represents a~special instance of the~threshold synthesis,~which tries to divide the~realisation set $\rlzf$ into two subsets based on their satisfiability.
We do not address this synthesis task in our work.
The second task on which we focus our attention is finding a~realisation that minimises or maximises a~given objective.

\begin{definition}[Feasibility]
Find a realisation $r \in \rlzf$ such that $\mathcal{P}[r] \models \varPhi$. 
\end{definition}

\begin{definition}[Minimality] \label{def:minimality}
For property $\phi_{\min}$, find a realisation $r^* \in \rlzf$ such that:
$$r^* \in \argmin_{r \in \rlzf} \left \{ \prob[\sketch[r] \models \phi_{\max}] \mid \sketch[r] \models \varPhi \right \}.$$.
\end{definition}

We defined only minimal synthesis task for probability property,~but its variants for maximisation and expected rewards may be defined analogously.
Moreover, we focus also to a relaxed variant of minimal synthesis, the so-called \textit{$\varepsilon$-minimal synthesis}, defined as follows: $\mathcal{P}[r^*] \models \varPhi \; and \; 
\mathbb{P}[\mathcal{P}[r^*] \models \varphi_{min}] \leq (1 - \varepsilon) \cdot \min_{r \in \mathcal{\overline{R}}} \{ \mathbb{P}[\mathcal{P}[r] \models \varphi_{min}] \; \lvert \; \mathcal{P}[r] \models \varPhi \}.$

\begin{example} (Synthesis problem)
Assume an~MCs family $\fml$ from Example~\ref{exam:mcfamily} and the~specification $\phi = \reachability{\geq}{0.1}{\{s_1\}}$.
The solution to the~feasibility synthesis problem is,~for example,~the~realisation $r_0$, since $D_{r_0}$ has a~probability of $\frac{2}{3}$ to reach state $s_1$.
For $\varPhi = [F \; \{s1\}]$,~the~solution to the~maximal synthesis problem on MCs family $\fml$ is the~realisation $r_2$,~as MC $D_{r_2}$ has a~probability equal to one to reach state $s_1$.
\end{example}

\section{Synthesis Methods}
Existing methods for a~probabilistic programs synthesis can be separated into two orthogonal classes.
The~first class involves complete methods that prove the~non-existence,~or potentially optimally,~of the~given probabilistic program.
On the~contrary,~incomplete methods handling various evolutionary techniques and intelligent search strategies form the~second one~\cite{spl2}.
However,~these incomplete methods cannot treat unfeasible and optimal synthesis problems compared to the~complete methods.
Therefore, we focus on the~complete state-of-the-art methods even though the~incomplete methods provide a~valuable flexibility level.
As a~reference and baseline method,~we consider a~\textit{one-by-one} approach that enumerates through each design space member~\cite{onebyone}.
An~explosion of the~design space renders this technique unfeasible for large families,~necessitating using advanced approaches utilising the~arbitrary structure of~the Markov chain family.

This thesis focuses on the~complete methods considered within the~\textit{oracle-guided} synthesis approach~\cite{oracle1,oracle2}.
The~control unit called \textit{learner} selects a~realisation $r$ from the~designs family and passes it to the~performing unit called \textit{oracle}.
This unit provides an~answer to whether realisation $r$ satisfies a given specification $\varPhi$.
Whenever it is not this case,~it provides supplementary information representing counterexample (CEGIS) or bounds from MC model checking (AR).
For purposes of this thesis, two various orthogonal oracles can be considered:
\begin{enumerate}[label=(\roman*)]
    \item \textbf{Inductive Oracle} (CE): It tries to infer the~declarations (counter-examples) about other family members by analysing individual realisation~\cite{cegis}.
    \item \textbf{Deductive Oracle} (AR): Abstraction refinement oracle considers more family members at once and then infers the consequences of these members' constructed aggregation~\cite{cegar}.
\end{enumerate}

\paragraph{Inductive Oracle.}
\textit{Counter-examples} represents a~concrete system execution violating a~given specification $\phi$.
When considering the~probabilistic model checking of Markov chains,~the~counter-examples represent paths set,~which have the~probabilities added to a~quantity that violates $\bowtie \lambda$.
When an~MC D violates the~given specification $\phi$ ($MC \; D \; \not\models \phi$),~a~ \textit{counter-example} provides diagnostic information related to the~states causing the~violation.
We consider the~counter-examples related to critical sub-systems:

\begin{definition}[Counter-example]
Let $D = \mc$ be an MC with $s_{\bot} \notin S$.
An MC $D \! \downarrow \! C = (S \cup \{ s_{\bot} \}, s_0, \mathbf{P}')$ induced by $C \subseteq S$ is sub-MC of MC $D$, where the transition probability matrix $\mathbf{P}'$ is defined as follows:
\begin{align*}
    \mathbf{P}'(s) = 
    \begin{cases}
        \mathbf{P}(S) \quad & if \; s \in C \\
        [ s_{\bot} \mapsto 1 ] \quad & \l otherwise.  
    \end{cases}
\end{align*}
For the~given specification $\reachability{\leq}{\lambda}{T}$,~when holds $D \! \downarrow \! C \not \models \reachability{\leq}{\lambda}{(T \cap (C \cup \{ s_0 \}))}$, then we call the~sub-system $D \! \downarrow \! C$ as a~\textit{counter-example} (CE).
\end{definition}

Let $D_r$ be an~MC that violates the~given specification $\phi$.
The inductive oracle constructs the critical sub-system $D_r \! \downarrow \! C$ to compute other realisations that also violate $\phi$.
Subsequently, it uses this constructed sub-system to infer a set called \textit{conflict} for $D_r$ and $\phi$.

\begin{definition}[Conflict] \label{def:conflict_set}
For MCs family $\fml = \family$,~and the~set $C \subseteq S$,~the~conflict set $K_C$ is given by $\bigcup_{s \in C} supp(\mathcal{B}(s))$ and consist of relevant parameters.
\end{definition}

\paragraph{Deductive Oracle.}
As we said above,~a~\textit{quotient} MDP over-approximates the~behaviours of each Markov chain in the~currently analysed family $\fml$.
The~\textit{deductive} oracle performs the~model checking of this MDP,~and it can yields the~following results.
We consider the~following specification against which is performed relevant model checking: $\reachability{\geq}{\lambda}{T}$.
The model checking itself returns the corresponding lower ($x_{min}$) and upper ($x_{max}$) bounds, as well as the minimising ($\sigma_{min}$) and maximising ($\sigma_{max}$) schedulers on the considered reachability probability.
When $x_{max} < \lambda$,~the~synthesis task is unfeasible since for each family member $r \in \rlz^\fml$ holds that $\mathcal{D}_r \not\models \phi$.
On the~contrary,~when $x_{min} >= \lambda$,~the~oracle can return an~arbitrary family member as a~solution to the~synthesis task since each satisfies $\phi$.
Last but not least,~when $x_{min} \leq \lambda \leq x_{max}$,~we cannot decide this case,~except when the~scheduler $\sigma_{max}$ is consistent.
Such scheduler represents the~valid member of the~analysed family with $x_{max} \geq \lambda$,~so it is a solution to the~synthesis task.
Otherwise,~when scheduler $\sigma_{max}$ is not consistent,~the~considered abstraction is too over-approximate,~and the~problem is undecided.

\begin{figure*}[h!]
\centering
\includegraphics[width=0.70\textwidth]{figures/ar_loop.pdf}
\caption{An \textit{Abstraction Refinement} (AR) analysing loop.}%
\label{fig:architecture}%
\end{figure*}


In such a~case, we split the~undecidable family of Markov chains into two derived subfamilies. 
Then each of them represents the~input for deductive oracle,~which analyse them using the~technique introduced above.
When derived subfamilies are still undecidable, we refine them into smaller and smaller subfamilies until we find a~feasible solution or explore the~whole family.
When the~sub-family represents concrete realisation,~so it induces an~MC and not MDP ($x_{min} = x_{max}$),~such family is necessarily decided and have not split.
Moreover,~since the~family size is finite,~the~presented technique has guaranteed the~termination for these two reasons.
 
\paragraph{TODO NAME.} The~novel integrated approach,~the~so-called \textit{hybrid} method,~combines both these oracles and can thus take advantage of the~benefits offered by both approaches~\cite{roman-DP,tacas21}.
We illustrate the~cooperation between the~individual units within this method in Figure~\ref{fig:adaptivesynt}.
The~learner unit maintains the~subfamilies queue $\rlz' \subseteq \rlzf$ for subsequent processing and decides which oracle is selected based on their previous efficiency.
The~\textit{CE-Oracle} analyses the~family member $r$ and,~when it satisfies the~given specification $\varPhi$,~then returns it as a~solution.
On the~other hand,~it can generalise the~analysed realisation $r$ into a~subfamily $\rlz'$,~and the~learner unit can discard it from the~whole design space $\rlzf$.
Moreover,~this oracle exploits the~MDP bounds when constructing counterexamples,~thanks to which it can generate more petite generalisations.
The~\textit{Abstr-Oracle} analyses a~given sub-family $\rlz \subseteq \rlzf$,~and according to the~result,~it performs the~subsequent action.
When all realisations $r \in \rlz$ satisfy $\varPhi$,~then it returns the~overall synthesis result as feasible.
In another case,~when all realisations violate $\varPhi$,~the~whole analysed sub-family $\rlz$ will be discarded from the~families queue $\rlzf$ by the~learner.
The last option returns safe bounds on the best- and worst-case behaviour of all realisations in $\rlz$ considered $\varPhi$ when the analysis result is inconclusive.

\begin{figure}[ht!]
    \centering
     \begin{tikzpicture}
    \node[rectangle, draw, inner sep=8pt] (learner) {Learner};
    \node[rectangle, draw, inner sep=8pt,right=2.7cm of learner] (oracle) {CE-Oracle};
       \node[rectangle, draw, inner sep=8pt,left=2.7cm of learner] (abst) {Abstr-Oracle};
    \node[above=0.5cm of learner] (rlz) {$\rlzf$};
    \draw[->] (rlz) -- (learner);
    \node[above=0.5cm of oracle] (phi) {$\varPhi$};
    \draw[->] (phi) -- (oracle);
    \node[above=0.5cm of abst] (phiab) {$\varPhi$};
    \draw[->] (phiab) -- (abst);
    \draw[->] (learner) edge[bend left=10] node[above] {\scriptsize{$r \in \rlzf$}+bounds} (oracle);
    \draw[->] (oracle) edge[bend left=10] node[below] {\scriptsize{$\rlz' \subseteq \rlzf$ violate $\varPhi$}} (learner);
      \draw[->] (learner) edge[bend right=10] node[above] {\scriptsize{$\rlz \subseteq \rlzf$}} (abst);
    \draw[->] (abst) edge[bend right=10] node[below,align=center] {\scriptsize{bounds \emph{or} $\rlz$ violates}} (learner);
    
    \node[below=0.4cm of oracle] (sat) {$r \models \varPhi$};
    \draw[->] (oracle) -- (sat);
    \node[below=0.4cm of abst] (allsat) {each $r \in \rlz$, $r \models \varPhi$};
    \draw[->] (abst) -- (allsat);
    \node[below=0.4cm of learner] (unsat) {no $r \models \varPhi$};
    \draw[->] (learner) -- (unsat);
    \end{tikzpicture}
  \vspace{-0,5em}
    \caption{Oracle-guided synthesis (adapted from~\cite{tacas21}).}
    \label{fig:adaptivesynt}.
    \vspace{-1em}
\end{figure}

\paragraph{Hybrid Method.}
An~extended synthesis approach was introduced in~\cite{roman-DP,tacas21} using the~abstraction refinement to family prune and accelerating the~construction of counter-examples by CE-oracle.
Its main idea is to perform a~limited number of abstraction refinement loops and then invoke CEGIS to one of the~refined sub-families.
It turned out that a~moment of the~switching can be crucial,~and therefore,~the~method has to detect the~rightest moment.
One AR iteration is typically significantly slower than one CEGIS iteration since the~AR iteration involves an~MDP model-checking,~which inspires whole method workflow.
The~advance of the~hybrid methods is also constructing counter-examples within the~CEGIS loop,~where it uses a~fast greedy approach providing smaller generalisations.

\begin{algorithm}[H]
\hspace*{\algorithmicindent} \textbf{Input:} A MCs family $\fml$, a reachability property $\varphi$. \\
\hspace*{\algorithmicindent} \textbf{Output:} $Realization\; r \in \rlzf \; s.t. \; \mathcal{D}_r \models \varphi$, otherwise UNSAT. \\
\vspace*{-1.5em}
\begin{algorithmic}[1]
    \STATE $\rlzf \leftarrow \{ \rlz^{\fml} \}$ \hfill \textbf{// each analysed (sub)-family also holds bounds}
    \STATE $\delta_{CEGIS} \leftarrow 1$ \hfill \textbf{// time allocation factor for CEGIS}
    \WHILE{$\rlzf \neq \emptyset$}
        \STATE result, $\rlzf'$, $\sigma_{AR}$, $t_{AR}$ $\leftarrow$ AR($\rlzf$, $\varphi$)
        \IF{\textbf{satisfiable}(result)}
            \RETURN result
        \ELSE
            \STATE result, $\rlzf''$, $\sigma_{CEGIS}$ $\leftarrow$ CEGIS($\rlzf'$, $\varphi$)
            \IF{\textbf{satisfiable}(result)}
                \RETURN result
            \ELSE
                \STATE $\delta_{CEGIS}$ $\leftarrow$ $\sigma_{CEGIS}$ / $\sigma_{AR}$
                \STATE $\rlzf$ $\leftarrow$ $\rlzf''$
            \ENDIF
        \ENDIF
    \ENDWHILE
    \RETURN{UNSAT}
\end{algorithmic}
\caption{Hybrid method: Feasibility synthesis with single property.}
\label{alg:hybrid}
\end{algorithm}

As we said above,~this \textit{novel integrated} method combines inductive and deductive oracles,~and we briefly summarise their operation.
CEGIS iterates through all family members (realisations) until it reached the~satisfying realisation,~if such exists,~or when it explores the~whole design space.
Moreover,~it constructs counter-examples whenever the~analysed realisation $r \in \fmlr$ unsatisfying the~given specification $\varPhi$.
Realisations subset $\rlz' \subseteq \rlzf$ represents such counter-examples that CEGIS subsequently prunes from the~analysed design space $\rlzf$.
On the~other hand,~an AR loop builds MDP models from the~sub-families queue $\rlz \subseteq \rlzf$ that model-checkers analyse in follows.
When the~analysis results are inconclusive,~an AR refines the~analysed sub-family and  stores the~obtained satisfiability bounds for further processing within CEGIS loop.
Method allocates the~time per both loop according to their performance,~e.g.,~it can consider the~number of pruned MCs per timed unit and estimates an~efficiency from it.
Namely,~when it notices that AR prunes sub-families twice as slow as CEGIS,~it increases time twice in the~next round for CEGIS.
The resulting algorithm is summarised in Algorithm~\ref{alg:hybrid},~adapted from~\cite{tacas21}.


\chapter{Advanced Methods for Probabilistic Synthesis}\label{chap:advanced}
The~developed framework for \textit{integrated} synthesis has been designed for \textit{feasibility} synthesis concerning a~\textit{single} property.
However,~probabilistic programs often have to satisfy specifications expressed as a~\textit{conjunction} of several temporal logic constraints,~potentially including the~\textit{optimal} objective.
Therefore,~we design extensions to generalise the~\textit{hybrid} method to handle \textit{multi-property} specifications and treat \textit{optimal} synthesis.
In the~following,~we introduce these extensions to adapt the~integrated synthesis for both advanced methods. 
We design them individually for both \emph{CEGIS} and \emph{AR} loop,~whereas the overall framework of the~hybrid method is unchanged.

\section{Multi-Property Synthesis}
When considering \textit{multi-property} specifications,~the basic ideas of both oracles (\textit{CEGIS} and \textit{AR}) remain the~same.
When \textit{AR} analyses the quotient MDP concerning multiple properties, it yields multiple probability bounds.
\textit{CEGIS} loop constructs a~separate conflict for each unsatisfied property whenever it meets an~unsatisfiable realisation.
Moreover,~it uses the~corresponding probability bounds obtained at the~\textit{AR} loop to improve the quality of generated counter-examples.

\paragraph{CEGIS.}
The~CEGIS performs the~\textit{multi-property} synthesis in the~almost same manner as the~\textit{feasibility} synthesis for a~single property,~but there are a~few differences.
We decided to analyse each property $\varphi_i \in \varPhi$ for each considered realisation $r \in \rlz$,~even if we come across a~property that the~given realisation does not satisfy.
We construct the~counter-examples whenever the~analysed realisation $r$ does not satisfy the~given specification $\varphi_{i}$.
In this way,~we~prune the~design space of the~analysed family more efficiently because each constructed counter-example throws out a~certain number of family members.
The core of the loop stays the same.
We pick the~concrete realisation $r \in \rlz$,~then construct the~corresponding MC $\mathcal{D}_r$ and perform the~model checking against to current specification $\varphi_i$.
The~synthesis terminates when a~satisfying realisation against the~whole specification set $\varPhi$ is found,~or the~whole state space is exhausted,~indicating that no feasible solution in the~analysed family exists.


\begin{algorithm}[h!]
\hspace*{\algorithmicindent} \textbf{Input:} A family $\fml$ of MCs with the set $\rlz \subseteq \rlzf$ of realisations, and a set of properties $\varPhi = \{ \varphi_0, \dots, \varphi_{N-1} \}$. \\
\hspace*{\algorithmicindent} \textbf{Output:}  A realisation $r \in \rlz$ such that $\forall \; 0 \leq i < N. \; \mathcal{D}_r \models \varphi_i$, if such exists, otherwise $\mathtt{UNSAT}$. \\
\vspace*{-1.5em}
\begin{algorithmic}[1]
    \WHILE{$\rlz \neq \emptyset$}
        \STATE $\mathtt{allSat} \leftarrow \mathtt{True}$
        \STATE $r \leftarrow \mathtt{any(\rlz)}$
        \STATE $\mathcal{D}_r\leftarrow \mathtt{buildDTMC(\mathcal{R}, r)}$
        \FOR{$\varphi_i \in \varPhi$}
            \STATE $\mathtt{sat} \leftarrow \mathtt{solveDTMC(\mathcal{D}_r, \varphi_{i})}$
            \IF{$\mathtt{sat}$}
                \IF{$i = N - 1 \; \wedge \; \mathtt{allSat}$}
                    \RETURN $r$
                \ELSE
                    \STATE $\mathtt{continue}$
                \ENDIF
            \ELSE
                \STATE $\mathcal{R} \leftarrow \mathcal{R} \setminus \mathtt{constructCE}(\mathcal{D}_r, \varphi_{i})$
                \STATE $\mathtt{allSat} \leftarrow \mathtt{False}$
            \ENDIF
        \ENDFOR
    \ENDWHILE
    \RETURN $\mathtt{UNSAT}$
\end{algorithmic}
\caption{CEGIS loop: Multi-property synthesis.}
\label{alg:cegis_multi}
\end{algorithm}

\paragraph{Abstraction Refinement.}
\textit{Multi-properties} synthesis within the~AR loop is performed in the~same loop as the~single-property synthesis,~with some modifications.
The~algorithm's input represents the~MCs family $\fml = \family$ with the~corresponding set of realisations $\rlz \subseteq \rlzf$,~and a~set of properties $\varPhi_D$ explicitly determined for this family $\fml$.
This specification set $\varPhi_\fml$ includes the~specifications that need to be analyzed within the~family,~in other words,~specifications that have not yet been satisfied.

The~algorithm first constructs the~\textit{quotient} MDP $M^\fml$ concerning the~given family $\fml$ and realisation set $R$,~and then iterates over sub-families of $U$ that have not been yet analysed.
For each selected sub-family $\fml$ is then performed the~MDP model checking on the~relevant restricted MDP,~which yields computed minimal ($\mathtt{min}$) and maximal ($\mathtt{max}$) probabilities,~and corresponding schedulers ($\sigma_{min}$ and $\sigma_{max}$) related to them,~respectively.
We analyse these obtained results concerning current specification,~mainly whether it is \textit{safety} or \textit{liveness} specification,~from their feasibility and decidability.

When holds $x_{min} > \lambda$ for safety property or $x_{max} < \lambda$ for liveness property ($\neg \mathtt{sat}$) then all realisations $r \in \rlz$ violate specification $\varphi_{i}$,~and the~algorithm continues on the~next sub-family or terminates with $\mathtt{UNSAT}$ result.
On the contrary,~when holds $x_{max} \leq \lambda$ for safety property or $x_{min} \geq \lambda$ for liveness property then each family member satisfy specification $\varphi_{i}$,~and~the~algorithm continues on the~next specification $\varphi_{i+1}$ or yields any realisation $r \in \rlz$ as the~solution to the~synthesis task.
Finally,~when holds $x_{min} \leq \lambda \leq x_{max}$ for both kinds of properties,~then we have to check whether the~corresponding scheduler $\sigma$ is consistent or not.
When it is consistent,~it represents a~family member satisfying the~specification $\varphi_{i}$,~since it has $x_{min} \leq \lambda$ for safety and $x_{max} \geq \lambda$ for liveness property.
Otherwise,~the~synthesis task is still undecided,~and the~currently analysed sub-family $\rlz$ will be split.

\begin{algorithm}[h!]
\hspace*{\algorithmicindent} \textbf{Input:} A family $\fml$ of MCs with the set $\rlz \subseteq \rlzf$ of realisations, and a properties set $\varPhi_{\fml} = \{\varphi_0, \dots, \varphi_M \}$. \\
\hspace*{\algorithmicindent} \textbf{Output:}  A realisation $r \in \rlz$ s.t. $\forall \; 0 \leq i < M. \; \mathcal{D}_r \models \varphi_i$, if such exists, otherwise $\mathtt{UNSAT}$. \\
\vspace*{-1.5em}
\begin{algorithmic}[1]
    \STATE $U \leftarrow \{ \rlz \}$
    \STATE $M^\fml \leftarrow \mathtt{buildQuotientMDP(\fml, \rlz)}$ \hfill \textbf{// Applying Definition 7 and 8 in~\cite{cegar}}
    \WHILE{$U \neq \emptyset$}
        % \STATE $\mathtt{allSat} \leftarrow \mathtt{True}$
        \STATE $\mathtt{select \; \rlz \in U}$, and $U \leftarrow U \setminus \{ \rlz \}$
        \STATE $M^\fml[\rlz] \leftarrow \mathtt{restrict(M^\fml, \rlz)}$ \hfill \textbf{// Applying Definition 12 in~\cite{cegar}}
        \FOR{$\varphi_i \in \varPhi_\fml$}
            \STATE $\sigma_{min}, \ min \leftarrow \mathtt{solveMinMDP(M^\fml[\rlz], \varphi_{i})}$
            \STATE $\sigma_{max}, \ max \leftarrow \mathtt{solveMaxMDP(M^\fml[\rlz], \varphi_{i})}$
            \STATE $\mathtt{sat}, \; \sigma \leftarrow \mathtt{checkResult}(\mathtt{min}, \, \mathtt{max}, \, \varphi_i)$
            \IF{$\mathtt{sat}$}
                \STATE $\mathbf{if} \; i = M-1 \; \mathbf{then} \; \mathbf{return} \; \mathtt{any(\rlz)} \; \mathbf{else} \; \mathtt{continue}$
            \ENDIF
            \IF{$\neg \mathtt{sat}$}
                \STATE $\mathbf{break}$
            \ENDIF
            \IF{$\exists \; r \in \rlz. \; \mathtt{isConsistentScheduler(\sigma)}$}
                \RETURN $r$
            \ENDIF
            \STATE $U \leftarrow U \; \cup \; \mathtt{split(\rlz, min, max, \sigma_{min}, \sigma_{max})}$ \hfill \textbf{// A comprehensive info in~\cite{cegar}} 
        \ENDFOR
    \ENDWHILE
    \RETURN{$\mathtt{UNSAT}$}
\end{algorithmic}
\caption{AR loop: Multi-property synthesis.}
\label{alg:ar_multi}
\end{algorithm}

\section{Optimal Synthesis}
\textit{Optimal} synthesis is designed similarly to the~\textit{feasibility} synthesis,~with one significant difference.
An~\textit{optimising} property represents the~given \textit{optimal} property,~and its threshold is updated whenever satisfactory realisation is found.
The goal is to exclude this solution from the~searched state space.
For instance,~this update translates to decreasing the~threshold of the minimising property when the~\textit{minimal} synthesis considered.
The~optimal synthesis yield the~optimal solution when all family members were explored.

\paragraph{CEGIS.}
The~algorithm to perform the~\textit{CEGIS} loop for minimal synthesis takes as the~input a~family of MCs $\fml = \family$,~represented with realisations set $\rlz \subseteq \rlzf$,~and the~given minimisation specification $\varphi_{min}$.
This approach synthesises a~realisation $r^* \in \rlz$ minimising the~satisfaction probability $\mathtt{min^*}$ of the~given minimal objective.
Initially,~the~algorithm sets the~corresponding threshold of minimal objective concerning its settings and pick any realisation $r \in \rlz$.
Subsequently,~it builds the~corresponding Markov chain $\mathcal{D}_r$ and performs the~model checking against to given objective $\varphi_{min}$.

We then check whether $\mathcal{D}_r \models \varphi_{min}$ and when yes,~we moreover check whether the~obtained quantitative $\mathtt{result}$ is less than currently found minimum $\mathtt{min^*}$.
We update the~current optimal realisation $\mathtt{r^*}$ and corresponding minimal value $\mathtt{min^*}$ according to the~currently analysed realisation in such a~situation.
Moreover,~we update the~threshold of the~\textit{minimising} property represented the~given minimal objective according to the~newest one.
We do this to not analyse for realisations worse than the~current one in subsequent iterations but to prune them by constructing counter-examples.

Otherwise,~when $\mathcal{D}_r \not\models \varPhi_{min}$ or the~current result is not better than the~current minimum,~we compute a~critical set $C$ for $\mathcal{D}_r$ and $\varphi_{min}$.
This critical set $C$ represents a~fragment of the~whole state space $S$ of the~analysed family $\fml$.
Subsequently,~the algorithm~constructs the~sub-system $\mathcal{D}\!\downarrow\!C$ as a~counter-example concerning computed critical set $C$.
We then subtract these constructed counter-examples from the~set $\fml$ of candidate solutions. 
We repeat this procedure until either the~whole state space is exhausted and at the~end returns the~found minimising realisation $\mathtt{r^*}$ and corresponding value $\mathtt{min^*}$.
This approach is summarised in Algorithm~\ref{alg:ar_optimal}.

\begin{algorithm}[h!]
\hspace*{\algorithmicindent} \textbf{Input:} A family $\fml$ of MCs with the set $\rlz \subseteq \rlzf$ of realisations, and a property $\varphi_{min}$. \\
\hspace*{\algorithmicindent} \textbf{Output:}  A realisation $r^* \in \rlz$ according to Definition~\ref{def:minimality}. \\
\vspace*{-1.5em}
\begin{algorithmic}[1]
    \STATE $\mathtt{min^*} \leftarrow 0$, $\mathtt{r^*} \leftarrow \emptyset$
    \STATE $\varphi_{min} \leftarrow \mathtt{setThreshold(min^*)}$
    \WHILE{$\rlz \neq \emptyset$}
        \STATE $r \leftarrow \mathtt{any(\rlz)}$
        \STATE $\mathcal{D}_r\leftarrow \mathtt{buildDTMC(\mathcal{R}, r)}$
        \STATE $result \leftarrow \mathtt{solveDTMC(\mathcal{D}_r, \varphi_{min})}$
        \IF{$result < min^*$}
            \STATE $\mathtt{r^*} \leftarrow r$, $\mathtt{min^*} \leftarrow min$
            \STATE $\varphi_{min} \leftarrow \mathtt{setThreshold(min^* - min^* \cdot \varepsilon)}$
        \ELSE
            \STATE $\mathcal{R} \leftarrow \mathcal{R} \setminus \mathtt{constructCE}(\mathcal{D}_r, \varphi_{min})$
        \ENDIF
    \ENDWHILE
    \RETURN $r*$, $min^*$
\end{algorithmic}
\caption{CEGIS loop: Minimality synthesis.}
\label{alg:cegis_optimal}
\end{algorithm}

\paragraph{Abstraction Refinement.}
We illustrate the~\textit{minimality} synthesis process within the AR loop by Algorithm~\ref{alg:ar_optimal}.
We remind that the~synthesis target is to find realisation $r^* \in \rlzf$ that minimises the~satisfaction probability $\mathtt{min^*}$ of the~minimal objective.
A~set $U$ serves to store sub-families of the~given family of realisations $\rlz$ that have not been yet analysed.
The~algorithm starts with building the~quotient MDP for the~whole analysed family $\fml$ concerning the~current realisations set $\rlz \subseteq \rlzf$.
Subsequently,~it restricts the~set of realisations to obtain the~corresponding sub-family for every $\rlz \in U$,~concerning the~quotient MDP $M^\fml$. 
Then,~the~algorithm continues with running the standard MDP model checking to compute the~minimal and maximal probability and corresponding schedulers,~respectively.

The~main difference between the~AR loop when feasibility synthesis is performed and now is the~interpretation of the~underlying MDP model checking results.
The~algorithm can discard $\rlz$ when the~$\mathtt{min}$ probability for $\rlz$ is above $\mathtt{min^*}$.
Otherwise,~it is required to check whether the~corresponding scheduler $\sigma_{min}$ is \textit{consistent} or not.
When it is consistent,~we can discard $\rlz$ and updated the current $\mathtt{min^*}$ and $\sigma_{min}$, respectively.
When the scheduler is not consistent but $\mathtt{max < min^*}$ is valid,~we can still update the~current $\mathtt{min^*}$ and enhance the~pruning process.
In such a~situation,~some realisations in $\rlz$ induce a~lower probability than current $\mathtt{min^*}$,~but actually,~we do not know which ones.
However,~when the~scheduler is not consistent,~regardless of whether $\mathtt{min^*}$ has been updated,~the~algorithm has to split the~analysed realisations set $\rlz$.
Consequently,~the~algorithm subsequently has to analyse the~derive sub-families since they can cover the~minimising realisation.

Whenever the~set $U$ is empty,~so all sub-families have been analysed,~the~algorithm terminates and yields the~found optimal realisation $r^*$.
This realisation is obtained by applying the~function \texttt{applyScheduler} concerning Definition~\ref{def:incuded_mc}.
The~termination is guaranteed since only a~finite number of subfamilies realisations has to be analysed.
We summarise this procedure in Algorithm~\ref{alg:ar_optimal}.

\begin{algorithm}[h!]
\hspace*{\algorithmicindent} \textbf{Input:} A family $\fml$ of MCs with the set $\rlz \subseteq \rlzf$ of realisations, and a property $\varphi_{min}$. \\
\hspace*{\algorithmicindent} \textbf{Output:}  A realisation $r^* \in \rlz$ according to Definition~\ref{def:minimality}. \\
\vspace*{-1.5em}
\begin{algorithmic}[1]
    \STATE $min^* \leftarrow 0$, $U \leftarrow \{ \rlz \}$
    \STATE $M^\fml \leftarrow \mathtt{buildQuotientMDP(\fml, \rlz)}$ \hfill \textbf{// Applying Definition 7 and 8 in~\cite{cegar}}
    \WHILE{$U \neq \emptyset$}
        \STATE $\mathtt{select \; \rlz \in U}$, and $U \leftarrow U \setminus \{ \rlz \}$
        \STATE $M^\fml[\rlz] \leftarrow \mathtt{restrict(M^\fml, \rlz)}$ \hfill \textbf{// Applying Definition 12 in~\cite{cegar}}
        \STATE $\sigma_{min}, \ min \leftarrow \mathtt{solveMinMDP(M^\fml[\rlz], \varphi_{min})}$
        \STATE $\sigma_{max}, \ max \leftarrow \mathtt{solveMaxMDP(M^\fml[\rlz], \varphi_{min})}$
        \IF{$min < min^*$}
% \hfill \textbf{// Applying Definition 10 in~\cite{cegar}}
            \IF{$\mathtt{isConsistentScheduler(\sigma_{min})}$}  
                \STATE $\sigma^* \leftarrow \sigma_{min}$, $min^* \leftarrow min$
            \ELSE
                \IF{$max < min^*$}
                    \STATE $min^* \leftarrow min$
                \ENDIF
                \STATE $U \leftarrow U \; \cup \; \mathtt{split(\rlz, min, max, \sigma_{min}, \sigma_{max})}$ \hfill \textbf{// A comprehensive info in~\cite{cegar}}
            \ENDIF
        \ENDIF
    \ENDWHILE
    \RETURN{$\mathtt{applyScheduler(\rlz, \sigma^*)}$}
\end{algorithmic}
\caption{AR loop: Minimality synthesis.}
\label{alg:ar_optimal}
\end{algorithm}


\chapter{Combined Probabilistic Synthesis}\label{chap:combined}

\textit{Parameter} synthesis problems targets finding parameter values for which the~parametric model (optimal) satisfies the~specification.
Parametric probabilistic models have several applications in a~wide range of areas.
For instance, in~\cite{ceska-biochemical} synthesised the~rate parameters in stochastic biochemical networks,~in~\cite{model-repair-usage} tuned the~model parameters exploiting the~parametric Markov chains.
Moreover,~parametric models have applications also when ranking the patches in the~software repair~\cite{learning-correct-code} and for computing perturbation bounds~\cite{nested-approximation}.

\section{Preliminaries}
We denote a~finite set of \textit{parameters} as $V$.
We consider \textit{parameters} over the~domain $\real$ ranged over by $x,y,z$ for the~following definitions.
Let $u: V \rightarrow \real$ denote a~\textit{valuation} for \textit{parameters} $V$.
We denote a~set of \textit{multi-affine multivariate polynomials} as $\mathbb{Q}_V$, where polynomial $f$ is defined over parameters $V$ and equal to $\sum_{i \leq M}{\prod_{v\in V_i}v \cdot a_i}$,~for appropriate $M \in \nat$, $a_i \in \rat$,~and $\forall \, 0 \leq i \leq m. \; V_i \subseteq V$.
We note that this set includes only the polynomials where the~variables have the~maximal degree equal to one,~i.e.,~$y^3 \notin \mathbb{Q}_v$~\cite{Quatmann2016}.

\begin{definition}[parametric Markov decision process]
\cite{Quatmann2016}
A parametric Markov decision process (pMDP) $\mathcal{M}$ is a tuple $(S, V, s_0, Act, \mathcal{P})$, where $S$ is a finite set of states, $V$ is a finite set of parameters over $\real$, $s_0 \in S$ is an initial state, $Act$ is a non-empty finite set of actions, and $\mathcal{P}: S \times Act  \times S \nrightarrow \mathbb{Q}_V$ is a transition probability function.
\end{definition}
A~set $\mathit{Act(s) = \{ a \in Act \; \lvert \; \exists s' \in S. \mathcal{P}(s, a, s') \neq \; \perp \}}$ for state $s \in S$ represents the~\textit{available} actions.
When holds $\lvert \mathit{Act(s)} \rvert = 1$ for each state $s \in S$,~such pMDP straightforward induces an~parametric Markov chain (pMC).

\section{Related Work}
The~synthesis tasks for parametric probabilistic programs can be divided into the~following two categories.

\paragraph{Topology Synthesis.}
This synthesis task considers a~finite set of parameters that affect the~\textit{topology} of Markov chains.
Finding a~family member that satisfies given specifications is \textit{NP}-complete in the~parameters' number and can be naively solved by analysing all individual family members~\cite{onebyone}.
An~alternative approach models the~family of MCs by an~MDP and uses off-the-shelf algorithms for MDP model-checking~\cite{allinone}.
Tools such as \emph{QFLan}~\cite{qflan} and \emph{ProFeat}~\cite{profeat} implements this approach to quantitatively analyse alternative designs of software product lines~\cite{sw-product-lines}.
These tools provide the~complete methods to solve synthesis tasks,~but they are limited strictly to small families,~so they do not scale.
Introduced inductive techniques,~motivated from this approach,~form on abstraction-refinement (\emph{AR}) over the MDP representation~\cite{cegar},~and counterexample guided inductive synthesis (\emph{CEGIS}) for MCs~\cite{cegis} have been proposed to achieve better scalability.
Experiments in previous papers~\cite{cegar,cegis} have shown that the~synthesis methods implemented in \emph{ProFeat} or \emph{QFLan} have evident deficits on the~benchmarks investigated in this paper.
This synthesis task is closely connected to controller synthesis for partially observed MDPs (POMPDs),~representing,~for instance,~a~famous Maze~\cite{maze} model for AI planning under uncertainty,~to which we will pay attention later.
Other techniques to solve the~controller synthesis for POMDPs also use neural network oracles to lead search~\cite{pomdp-2} and adaptive learning schemes based on imitation learning~\cite{POMDP3}.
% This fact is supported by the comparison of the hybrid approach with the~one-by-one enumeration, we present in Section 5.

\paragraph{Parameter Synthesis.}
It considers models with a~fixed topology but with uncertain parameters associated with transition probabilities (or rates).
It analyses how the~parameter values affect the~behaviour of Markov chains (or MDPs).
The~state-of-the-art probability model checkers \storm{}~\cite{STORM} and \prism{}~\cite{KNP11} provide approximate techniques for parameter synthesis that solve the~same parameters in various transitions independently.
We note that these model checkers and others with the~same focus are primarily determined to verify concrete MC and not to the~whole MCs family.
On the contrary,~exact techniques construct rational functions for symbolic reachability probabilities proposed in~\cite{DBLP:conf/ictac/Daws04} and further improved in~\cite{dehnert2015prophesy}.
The~application of this synthesis task can be found, for instance,~in the~field of model repair problems~\cite{model-repair-1}.


\chapter{\toolname{}}\label{chap:paynt}

\toolname{}\footnote{\url{https://github.com/gargantophob/synthesis}} (\underline{P}robabilistic progr\underline{A}m s\underline{YNT}hesiser) is a~tool to synthesise probabilistic programs automatically.
\toolname{} supports the~synthesis of finite-state programs and the~specifications given as a~conjunction of temporal logic constraints,~possibly including an~optimal objective.
The input is a~program sketch that concisely describes a~finite family of (finite) Markov chains,~and a~specification.
The tool can then identify a~family member that (potentially optimally) satisfies given specifications.
Figure~\ref{fig:sketching} depicts the workflow of \toolname{}.

\begin{figure*}[h!]
\centering
\includegraphics[width=1.0\textwidth]{figures/sketching}
\caption{The workflow of the synthesis process within \toolname{}.}%
\label{fig:sketching}%
\end{figure*}

\section{Architecture}

The design of \toolname{} is based on an~oracle-guided synthesis~\cite{tacas21} enabling a~flexible combination and integration of a~variety of state-of-the-art synthesis and verification algorithms. 
In particular,~it implements a~hybrid synthesis approach leveraging both the~\emph{CEGIS} and the~\emph{AR} oracle. 
\toolname{} is able to efficiently synthesise the~program topology (\emph{topology} synthesis) as well as continuous parameters affecting the~transition probabilities (\emph{parameter} synthesis).
Moreover,~it can handle sketches including both types of synthesis problems,~so-called a~\emph{combined} synthesis,~which is a~unique feature compering to the~existing tools. 

To achieve a high-performance synthesis, \toolname{} is implemented on top of the~probabilistic model chec\-ker \storm{}~\cite{STORM},~providing optimised verification procedures.
It is implemented in a~modular fashion on top of a~python API to provide flexibility.
\toolname{} allows to define of the~program sketches and provides all baseline synthesis algorithms under one roof.
The~tool finds application primarily for two groups of users.
First,~the~analysis of realisations sets is a~useful back-end for automated system design. %automatic engines.
For instance,~this can be used when synthesising finite-state controllers for partially observable Markov decision processes (POMDPs),~synthesising a~network protocol to increase the~packet throughput,~or selecting the~optimal power management strategy.
Second,~\toolname{} provides a modular development platform for automated design of probabilistic programs.

\begin{figure*}[h!]
\centering
\includegraphics[width=1.0\textwidth]{figures/architecture.pdf}
\caption{The architecture of tool \toolname{}.}%
\label{fig:architecture}%
\end{figure*}

\toolname{}'s architecture (see Figure~\ref{fig:architecture}) consists of model checkers,~modules to build models and components for family handling.
The~family handlers store the~information about the~already covered design space of the~analysed family.
As the~name suggests,~the~member enumeration unit iterates over all family members.
SAT solver maintains an~SAT-formula describing undiscovered realisations,~and subsequent,~it uses Z3 SMT-solver to obtain the next candidate realisation.
The~queue with sub-families contains a~collection of unexplored sub-families refined as hyper-rectangles when the~analysis is inconclusive.
The~model builders take a~specific input according to the~active oracle and produce the~relevant model representation:
the~CE-oracle sends to the builder a single realisation $r$ while 
the~AR-oracle sends a~realisations set $\rlz' \subseteq \rlz$.
The~model checkers verify whether the constructed model (MCs or MDPs) satisfies the~given specification.
When analysing MDP,~lower and upper bounds on satisfiability probabilities are provided.
Last but not least,~\toolname{} provides a~module for generating counterexamples. In particular,~it implements two approaches: a~greedy state-expansion and a~MaxSat approach.

\paragraph{Implementation Frame.}
\toolname{} takes as the~input a~sketch written in the~\jani{} or \prism{} language and a~set of temporal properties expressed using the~PRISM syntax. 
\toolname{} is implemented on top of a~modern probabilistic model checker \storm{}~\cite{STORM} providing  high-performance verification procedures implemented in C++.
Further,~it uses Z3 theorem prover for SMT-solving and~a Python API for flexible implementation of the~synthesis loop itself.
The~implementation of \toolname{} is composed of \textit{30} Python modules containing \textit{7k} source lines of code.
We consider only our implementation and do not include extensions contributed to \storm{} and its Python API,~invoked by \toolname{}.
We tests the specific components with unit tests to maintain their correct functionality.
Regression tests verify the~accuracy and correctness of the~synthesis results and these tests currently cover more than \textit{90\%} of the~source code lines.

\section{\prism{} Sketch Language}
As we said above,~\toolname{} takes as the~input the~sketch written in the~\prism{}~\cite{KNP11} or \jani{}~\cite{jani} language.
These high-level programming languages to describe probabilistic systems are more advantageous than modelling the~system as a~Markov Chain.
The~state explosion problem,~arising when using the~Markov chains as an~operational model,~renders this approach unusable.
\storm{} parses the~high-level description (sketch) written in these languages and constructs the~corresponding Markov chain,~which the~individual synthesis methods analyse.
Now,~we briefly introduce the~\prism{} sketching language proposed in~\cite{cegis}.

A~program written in \prism{} language consists of modules that interact with between itself.
We consider only programs with a~single module since more modules can be transformed into one model program.
A~module state space is given by the~set of (bounded) variables,~with their initial values and the~set of guarded commands describing the~transitions between module states.
The commands have the following form:
\begin{align*}
\texttt{[action]} \
\texttt{guard}
\ \ \rightarrow \ \
p_1 : \texttt{update}_1 + \dots + p_n : \texttt{update}_n 
\end{align*}
The~\emph{actions} ensure the~synchronisation between two or more modules when they perform the~command.
The~\emph{guard} represents a~boolean expression over the~module variables.
An~update of the~variables is selected concerning the~probability distribution defined by expressions $p_1$ through $p_n$ when guard evaluates to true.
Essentially,~the guard identifies states for which this command is applicable,~and updates describe successor states and the~probability distribution over these successors.
Overlapping of guards is disallowed since it yields non-determinism.

Moreover,~a~\textit{sketch} describes a~program containing holes representing undefined program parts that should be filled with the~value from the~finite options set.
These holes are declared as follows:
\begin{align*}
    \texttt{type} \ \texttt{hole} \ \textit{h} \ \texttt{in} \ \ \{ \mathtt{expr_1}, \dots, \mathtt{expr_k} \}
\end{align*}
The~\texttt{type} represents the domain of \texttt{hole} \textit{h} and it is an \texttt{integer} or \texttt{float}.
The~hole identifier is given by \textit{h},~and the~set of expressions $\mathtt{expr_i}$ is defined over the~program variables.
Commands and variable declarations use a~\texttt{hole} as a~component of an~\texttt{update expression} or a~\texttt{guard}.
Such a~program sketch represents a~system description with a~specified general structure but with some concrete details left out.
Instantiating all holes yields a~specific program,~and the~synthesiser target is to resolve how to substitute all holes with their options to satisfy a~given specification.
\prism{} sketching language also provides several extra functionalities: specification of the~constraints to hole values and assigning costs to option holes,~and many others.
For more detailed information,~please refer to~\cite{cegis}.

\begin{example}[\prism{} Program]
For instance,~we introduce the~following \prism{} program $\sketch$:
\begin{align*}
    & \texttt{int} \ \texttt{hole} \ \mathit{X} \ \texttt{in} \ \{ 0,1 \} \\
    & \texttt{int} \ \texttt{hole} \ \mathit{Y} \ \texttt{in} \ \{ 1,2,3 \} \\
    & \texttt{module m} \\
    & \quad \texttt{int} \ \texttt{s:} \; [0..4] \; \texttt{init} \ \texttt{X+1} \\
    & \quad [] \ \texttt{s < Y} \ \rightarrow \ \texttt{0.75}: \ \texttt{(s' = max(s-X, 0))} \ \texttt{+} \ \texttt{0.25}: \ \texttt{(s' = s+X)}; \\
    & \quad [] \ \texttt{s = Y} \ \rightarrow \ \texttt{0.50}: \ \texttt{(s' = s-1)} \ \texttt{+} \ \texttt{0.50}: \ \texttt{(s' = s+1)}; \\
    & \quad [] \ \texttt{s > Y} \ \rightarrow \ \texttt{0.25}: \ \texttt{(s' = s-1)} \ \texttt{+} \ \texttt{0.75}: \ \texttt{(s' = min(s+1, 4))};  \\
    & \texttt{end module}
\end{align*}
\label{exa:prism}
\end{example}
The~module \texttt{m} can be in one of five states: $\{ 0,1,2,3,4 \}$.
When the~module state \texttt{s} is equal to the~value of hole \texttt{Y} (\texttt{s = Y}),~it will move to the~nearest previous (\texttt{s' = s-1}) or next (\texttt{s' = s+1}) state with the~same probability of \texttt{0.50}.
When the~module state \texttt{s} is above the~value of hole \texttt{Y} (\texttt{s > Y}),~it will move to the~left neighbour with the~probability of \texttt{0.75} and with remaining to the~right if it exists,~or stay in the~last state (\texttt{s=4}).
Otherwise (\texttt{s < Y}),~the model will move the~same as in the~previous case (\texttt{X=1}),~or it stays in the~current state without change (\texttt{X=0}).
When we consider the~following instantiating of the~holes \texttt{X=1} and \texttt{Y=2},~the~corresponding underlying Markov chain is depicted below.

\begin{figure*}[h!]
\centering
\includegraphics[width=1.0\textwidth]{figures/prism_dtmc.pdf}
\caption{An underlying Markov Chain of a probabilistic program $\sketch$ from Example~\ref{exa:prism}.}%
\label{fig:architecture}%
\end{figure*}

\section{Usage of \toolname{}}
We demonstrate the~usage of \toolname{} on a~synthesis problem considering a simple server for request processing depicted in Figure~\ref{fig:dpm}. 
Requests are produced by an~external unit within random intervals and maintained in a~request queue with capacity $Q_{max}$.
If the~queue is full arriving requests are lost.
The~server can operate in three modes having a different power consumption \,--\, \textit{active},~\textit{idle} and \textit{sleeping}.
The~server process the~requests only in the~\textit{active} state.
When the~server switches from a~state with low energy into a~state with higher,~extra energy is consumed,~and random latency is required.
We note that the~power consumption of request processing depends on the~current size of the~queue.
The~server operation time is finite but given as a~random process.

\begin{figure*}[h!]
\centering
\includegraphics[width=0.9\textwidth]{figures/dpm.pdf}
\caption{The server for request processing within \emph{DPM} case study.}%
\label{fig:dpm}%
\end{figure*}

The~synthesis aims to construct a~unit called \textit{power manager} (PM),~which controls the~server.
PM observes the~current queue size and,~according to it,~then sets the~relevant power profile.
Precisely,~it differentiates between four queue occupancy levels determined by the~threshold levels $T_1,T_2$,~and $T_3$.
They indicate which fraction of the~queue capacity is currently occupied,~and they are entered into the~model as unknown parameters.
Since the~model considers three levels,~then the~power manager observers the~queue occupancy on the~following intervals: $\left[0, T_1 \right]$, $\left(T_1, T_2 \right]$, $\left(T_2, T_3 \right]$, $\left(T_3, 1 \right)$.
Moreover,~it considers a~single power profile $P_1,\dots,P_4 \in \{0,1,2\}$ for each occupancy level.
The~power profile's current value represents the~server's mode,~so the set~$\{0,1,2\}$ encodes available modes sleeping,~idle and active in the~given order.
The~following sketch describes the~module of \textit{PM}:

\begin{verbatim}
module PM
    pm  :  [0..2] init 0; // 0 - sleep, 1 - idle, 2 - active
    [sync0] q <= T1*QMAX -> (pm'=P1);
    [sync0] q > T1*QMAX & q <= T2*QMAX -> (pm'=P2);
    [sync0] q > T2*QMAX & q <= T3*QMAX -> (pm'=P3);
    [sync0] q > T3*QMAX -> (pm'=P4);
endmodule
\end{verbatim}

In this model,~we consider the~following parameters and their domains: the~queue capacity $Q_{\max} \in \{1,2,\dots,10\}$,~the~power profiles $P_1,\dots,P_4 \in \{0,1,2\}$ and the~thresholds $T_1 \in \{0.0,0.1,0.2,0.3,0.4\}$,~$T_2 \in \{0.5\}$,~$T_3 \in \{0.6,0.7,0.8,0.9\}$.
The~final sketch describing this considered model forms a~design space of $16,200$ various power managers.
The~average size of the Markov chains in this models is around $900$ states.
The~synthesis target is to find the~specific power manager,~i.e.,~the~holes instantiation,~that minimises power consumption while the~expected number of lost requests during the~server's operation time is below $1$.
We can formalise these requirements as a~pair of temporal logic formulae in the~\prism{} language:
\begin{verbatim}
R{"lost"} <= 1 [ F "finished" ]  
R{"power"}min=? [ F "finished" ]
\end{verbatim}

\toolname{} explores the~design space and produces the~following output,~including the~parameter assignment,~and the~quality value corresponds to $\Phi$ of the~analysed program:
\begin{verbatim}
QMAX=5,T1=0,T3=0.7,P1=1,P2=2,P3=2,P4=2
R[exp]{"lost"}=0.6823 [F "finished"]
R[exp]{"power"}min=9100 [F "finished"]
\end{verbatim}
The~synthesised power manager,~optimal wrt. to the~specified properties,~has the~queue capacity set to $5$ and individual thresholds at values  $1 = 0.0$, $2 = \floor{ 5\cdot 0.5}$ and $3 = \floor{ 5\cdot 0.7}$.
The~synthesised power manager performs the following strategy.
When the~request queue is empty,~then the~power manager maintains an~idle state.
Otherwise,~it always maintains an~active state,~regardless of the~exact size of the~queue,~and is never in the~sleeping state.
The~synthesised solution has a~power consumption of $9,100$ units and the~expected number of lost requests of $\approx 0.68 < 1$.

\toolname{} computes an~optimal solution in one minute. 
Although this synthesis problem is quite simple (recall it includes only 16k candidates),~\toolname{} is already $3 \times$ faster than a~naive enumeration of all realisations.
Further,~we explored a~more complex variant of this problem inspired by a~dynamical power manager's known model for complex electronic systems.
The~synthesis problem is described by the~sketch,~which covers a~family around 43M realisations.
\toolname{} solves this synthesis problem within 10~hours,~whereas the~naive enumeration takes more than one~month.

\chapter{Experimental Evaluation}\label{chap:experiments}


\chapter{Final Considerations}\label{chap:conclusion}


\section{Future Research}
\section{Conclusions}
  \else
    \input{projekt-01-kapitoly-chapters}
  \fi
  
  % Kompilace po částech (viz výše, nutno odkomentovat)
  % Compilation piecewise (see above, it is necessary to uncomment it)
  %\subfile{projekt-01-uvod-introduction}
  % ...
  %\subfile{chapters/projekt-05-conclusion}


  % Pouzita literatura / Bibliography
  % ----------------------------------------------
\ifslovak
  \makeatletter
  \def\@openbib@code{\addcontentsline{toc}{chapter}{Literatúra}}
  \makeatother
  \bibliographystyle{bib-styles/Pysny/skplain}
\else
  \ifczech
    \makeatletter
    \def\@openbib@code{\addcontentsline{toc}{chapter}{Literatura}}
    \makeatother
    \bibliographystyle{bib-styles/Pysny/czplain}
  \else 
    \makeatletter
    \def\@openbib@code{\addcontentsline{toc}{chapter}{Bibliography}}
    \makeatother
    \bibliographystyle{bib-styles/Pysny/enplain}
  %  \bibliographystyle{alpha}
  \fi
\fi
  \begin{flushleft}
  \bibliography{projekt-20-literatura-bibliography}
  \end{flushleft}

  % vynechani stranky v oboustrannem rezimu
  % Skip the page in the two-sided mode
  \iftwoside
    \cleardoublepage
  \fi

  % Prilohy / Appendices
  % ---------------------------------------------
  \appendix
\ifczech
  \renewcommand{\appendixpagename}{Přílohy}
  \renewcommand{\appendixtocname}{Přílohy}
  \renewcommand{\appendixname}{Příloha}
\fi
\ifslovak
  \renewcommand{\appendixpagename}{Prílohy}
  \renewcommand{\appendixtocname}{Prílohy}
  \renewcommand{\appendixname}{Príloha}
\fi
%  \appendixpage

% vynechani stranky v oboustrannem rezimu
% Skip the page in the two-sided mode
%\iftwoside
%  \cleardoublepage
%\fi
  
\ifslovak
%  \section*{Zoznam príloh}
%  \addcontentsline{toc}{section}{Zoznam príloh}
\else
  \ifczech
%    \section*{Seznam příloh}
%    \addcontentsline{toc}{section}{Seznam příloh}
  \else
%    \section*{List of Appendices}
%    \addcontentsline{toc}{section}{List of Appendices}
  \fi
\fi
  \startcontents[chapters]
  \setlength{\parskip}{0pt} 
  % seznam příloh / list of appendices
  % \printcontents[chapters]{l}{0}{\setcounter{tocdepth}{2}}
  
  \ifODSAZ
    \setlength{\parskip}{0.5\bigskipamount}
  \else
    \setlength{\parskip}{0pt}
  \fi
  
  % vynechani stranky v oboustrannem rezimu
  \iftwoside
    \cleardoublepage
  \fi
  
  % Přílohy / Appendices
  \ifenglish
    \chapter{Storage Medium}
\begin{itemize}
   \item[] \texttt{/synthesis/*}\,---\,source code of \toolname{} from date May 25, 2021
   \item[] \texttt{/README.txt}\,---\,useful information about the storage medium content
   \item[] \texttt{/text/*}\,---\,source code of this thesis
   \item[] \texttt{/xstupi00.pdf}\,---\,final version of this thesis
\end{itemize}
  \else
    \input{projekt-30-prilohy-appendices}
  \fi
  
  % Kompilace po částech (viz výše, nutno odkomentovat)
  % Compilation piecewise (see above, it is necessary to uncomment it)
  %\subfile{projekt-30-prilohy-appendices}
  
\end{document}

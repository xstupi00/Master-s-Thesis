%==============================================================================
% tento soubor pouzijte jako zaklad
% this file should be used as a base for the thesis
% Autoři / Authors: 2008 Michal Bidlo, 2019 Jaroslav Dytrych
% Kontakt pro dotazy a připomínky: sablona@fit.vutbr.cz
% Contact for questions and comments: sablona@fit.vutbr.cz
%==============================================================================
% kodovani: UTF-8 (zmena prikazem iconv, recode nebo cstocs)
% encoding: UTF-8 (you can change it by command iconv, recode or cstocs)
%------------------------------------------------------------------------------
% zpracování / processing: make, make pdf, make clean
%==============================================================================
% Soubory, které je nutné upravit nebo smazat: / Files which have to be edited or deleted:
%   projekt-20-literatura-bibliography.bib - literatura / bibliography
%   projekt-01-kapitoly-chapters.tex - obsah práce / the thesis content
%   projekt-01-kapitoly-chapters-en.tex - obsah práce v angličtině / the thesis content in English
%   projekt-30-prilohy-appendices.tex - přílohy / appendices
%   projekt-30-prilohy-appendices-en.tex - přílohy v angličtině / appendices in English
%==============================================================================
% \documentclass[]{fitthesis} % bez zadání - pro začátek práce, aby nebyl problém s překladem
\documentclass[english]{fitthesis} % without assignment - for the work start to avoid compilation problem
%\documentclass[zadani]{fitthesis} % odevzdani do wisu a/nebo tisk s barevnými odkazy - odkazy jsou barevné
% \documentclass[english,enslovak,zadani]{fitthesis} % for submission to the IS FIT and/or print with color links - links are color
%\documentclass[zadani,print]{fitthesis} % pro černobílý tisk - odkazy jsou černé
%\documentclass[english,zadani,print]{fitthesis} % for the black and white print - links are black
%\documentclass[zadani,cprint]{fitthesis} % pro barevný tisk - odkazy jsou černé, znak VUT barevný
%\documentclass[english,zadani,cprint]{fitthesis} % for the print - links are black, logo is color
% * Je-li práce psaná v anglickém jazyce, je zapotřebí u třídy použít 
%   parametr english následovně:
%   If thesis is written in English, it is necessary to use 
%   parameter english as follows:
%      \documentclass[english]{fitthesis}
% * Je-li práce psaná ve slovenském jazyce, je zapotřebí u třídy použít 
%   parametr slovak následovně:
%   If the work is written in the Slovak language, it is necessary 
%   to use parameter slovak as follows:
%      \documentclass[slovak]{fitthesis}
% * Je-li práce psaná v anglickém jazyce se slovenským abstraktem apod., 
%   je zapotřebí u třídy použít parametry english a enslovak následovně:
%   If the work is written in English with the Slovak abstract, etc., 
%   it is necessary to use parameters english and enslovak as follows:
%      \documentclass[english,enslovak]{fitthesis}

% Základní balíčky jsou dole v souboru šablony fitthesis.cls
% Basic packages are at the bottom of template file fitthesis.cls
% zde můžeme vložit vlastní balíčky / you can place own packages here

%PAYNT (Probabilistic progrAm sYNThesizer)
\newcommand{\toolname}{\textsc{PAYNT}}
\newcommand{\storm}{\textsc{Storm}}
\newcommand{\prism}{\textsc{Prism}}

% numbers
\newcommand{\nat}{\mathbb{N}_0}
\newcommand{\rat}{\mathbb{Q}}
\newcommand{\real}{\mathbb{R}}
\newcommand{\realpos}{\real_{\geq 0}}
\newcommand{\unitinterval}{[0,1]}

% vectors
\newcommand{\vect}[1]{\boldsymbol{#1}}
\newcommand{\norm}[1]{\left\| #1 \right\|_1}

% probability
\newcommand{\prob}{\mathbb{P}}
\newcommand{\rew}{\mathbb{E}}
\newcommand{\supp}{\mathrm{supp}}

% properties
\renewcommand{\phi}{\varphi}
% \newcommand{\F}[1]{\Diamond#1}
\newcommand{\F}[1]{\mathrm{F}\ #1}
\newcommand{\reachability}[3]{\prob_{#1 #2}[\F{#3}]}
\newcommand{\reward}[3]{\rew_{#1 #2}[\F{#3}]}
\newcommand{\safety}[2]{\reachability{\leq}{#1}{#2}}
\newcommand{\liveness}[2]{\reachability{\geq}{#1}{#2}}
\newcommand{\safetys}{\safety{\lambda}{T}}
\newcommand{\livenesss}{\liveness{\lambda}{T}}
\DeclareMathOperator*{\argmax}{arg\,max}
\newcommand{\sketch}{\mathrm{P}}

% transition systems
\newcommand{\sins}{s \in S}
\newcommand{\sinit}{s_{0}}
\newcommand{\sBot}{s_{\bot}}
\newcommand{\sTop}{s_{\top}}
\newcommand{\statesum}[1]{\sum_{#1 \in S}}
\newcommand{\lastpi}{\mathrm{last}(\pi)}
\newcommand{\paths}[1]{\mathrm{Paths}^{#1}}
\newcommand{\pathsfin}[1]{\mathrm{Paths}^{#1}_{\mathrm{fin}}}

% mc
\renewcommand{\pm}{\vect{P}}
\newcommand{\mc}{(S,\sinit,\pm)}
\newcommand{\subchain}[2]{#1 {\downarrow} #2}

% mdp
\newcommand{\mpm}{\mathcal{P}}
\newcommand{\mdp}{(S,\sinit,Act,\mpm)}
\newcommand{\sigmamin}{\sigma_{\min}}
\newcommand{\sigmamax}{\sigma_{\max}}

% other
\newcommand{\vz}{\vect{0}}
\newcommand{\vo}{\vect{1}}

\newcommand{\vx}{\vect{x}}
\newcommand{\vxmin}{\vect{x}_{\min}}
\newcommand{\vxmax}{\vect{x}_{\max}}
\newcommand{\vxi}{\vx^{(i)}}
\newcommand{\vg}{\vect{\gamma}}
\newcommand{\vc}{\vect{\chi}}
\newcommand{\vu}{\vect{\upsilon}}

% family
\newcommand{\fml}{\mathcal{D}}
\newcommand{\fpm}{\mathcal{B}}
\newcommand{\family}{(S,\sinit,K,\fpm)}

% realization
\newcommand{\rlz}{\mathcal{R}}
\newcommand{\rlzf}{\overline{\rlz}}
\newcommand{\fmlr}{\fml_r}

% CEGIS, CEGAR
\newcommand{\generalize}[2]{#1 {\uparrow} #2}
\newcommand{\generalizes}{\generalize{r}{\overline{K}}}
\newcommand{\quotientmdp}{(S,\sinit,\rlzf,\mpm)}

% algorithms
\newcommand{\successRate}[1]{\sigma_{#1}}

% Integrated: parameters
\newcommand{\mml}{mdpMClimit}
\newcommand{\api}{activatePerIter}

\usepackage{amsthm}

\theoremstyle{definition}
\newtheorem{definition}{Definition}
\newtheorem{example}{Example}


% Kompilace po částech (rychlejší, ale v náhledu nemusí být vše aktuální)
% Compilation piecewise (faster, but not all parts in preview will be up-to-date)
% \usepackage{subfiles}

% Nastavení cesty k obrázkům
% Setting of a path to the pictures
%\graphicspath{{obrazky-figures/}{./obrazky-figures/}}
%\graphicspath{{obrazky-figures/}{../obrazky-figures/}}

%---rm---------------
\renewcommand{\rmdefault}{lmr}%zavede Latin Modern Roman jako rm / set Latin Modern Roman as rm
%---sf---------------
\renewcommand{\sfdefault}{qhv}%zavede TeX Gyre Heros jako sf
%---tt------------
\renewcommand{\ttdefault}{lmtt}% zavede Latin Modern tt jako tt

% vypne funkci šablony, která automaticky nahrazuje uvozovky,
% aby nebyly prováděny nevhodné náhrady v popisech API apod.
% disables function of the template which replaces quotation marks
% to avoid unnecessary replacements in the API descriptions etc.
\csdoublequotesoff


\usepackage{tikz}
\usetikzlibrary{automata,positioning,shapes,fit,calc}
\usepackage{url}


% =======================================================================
% balíček "hyperref" vytváří klikací odkazy v pdf, pokud tedy použijeme pdflatex
% problém je, že balíček hyperref musí být uveden jako poslední, takže nemůže
% být v šabloně
% "hyperref" package create clickable links in pdf if you are using pdflatex.
% Problem is that this package have to be introduced as the last one so it 
% can not be placed in the template file.
\ifWis
\ifx\pdfoutput\undefined % nejedeme pod pdflatexem / we are not using pdflatex
\else
  \usepackage{color}
  \usepackage[unicode,colorlinks,hyperindex,plainpages=false,pdftex]{hyperref}
  \definecolor{hrcolor-ref}{RGB}{223,52,30}
  \definecolor{hrcolor-cite}{HTML}{2F8F00}
  \definecolor{hrcolor-urls}{HTML}{092EAB}
  \hypersetup{
	linkcolor=hrcolor-ref,
	citecolor=hrcolor-cite,
	filecolor=magenta,
	urlcolor=hrcolor-urls
  }
  \def\pdfBorderAttrs{/Border [0 0 0] }  % bez okrajů kolem odkazů / without margins around links
  \pdfcompresslevel=9
\fi
\else % pro tisk budou odkazy, na které se dá klikat, černé / for the print clickable links will be black
\ifx\pdfoutput\undefined % nejedeme pod pdflatexem / we are not using pdflatex
\else
  \usepackage{color}
  \usepackage[unicode,colorlinks,hyperindex,plainpages=false,pdftex,urlcolor=black,linkcolor=black,citecolor=black]{hyperref}
  \definecolor{links}{rgb}{0,0,0}
  \definecolor{anchors}{rgb}{0,0,0}
  \def\AnchorColor{anchors}
  \def\LinkColor{links}
  \def\pdfBorderAttrs{/Border [0 0 0] } % bez okrajů kolem odkazů / without margins around links
  \pdfcompresslevel=9
\fi
\fi
% Řešení problému, kdy klikací odkazy na obrázky vedou za obrázek
% This solves the problems with links which leads after the picture
\usepackage[all]{hypcap}

% Informace o práci/projektu / Information about the thesis
%---------------------------------------------------------------------------
\projectinfo{
  %Prace / Thesis
  project={DP},            %typ práce BP/SP/DP/DR  / thesis type (SP = term project)
  year={2021},             % rok odevzdání / year of submission
  date=\today,             % datum odevzdání / submission date
  %Nazev prace / thesis title
  title.cs={Pokročilé metódy pre syntézu pravdepodobnostných programov},  % název práce v češtině či slovenštině (dle zadání) / thesis title in czech language (according to assignment)
  title.en={Advanced Methods for Synthesis of Probabilistic Programs}, % název práce v angličtině / thesis title in english
  %title.length={14.5cm}, % nastavení délky bloku s titulkem pro úpravu zalomení řádku (lze definovat zde nebo níže) / setting the length of a block with a thesis title for adjusting a line break (can be defined here or below)
  %sectitle.length={14.5cm}, % nastavení délky bloku s druhým titulkem pro úpravu zalomení řádku (lze definovat zde nebo níže) / setting the length of a block with a second thesis title for adjusting a line break (can be defined here or below)
  %Autor / Author
  author.name={Šimon},   % jméno autora / author name
  author.surname={Stupinský},   % příjmení autora / author surname 
  author.title.p={Bc.}, % titul před jménem (nepovinné) / title before the name (optional)
  %author.title.a={Ph.D.}, % titul za jménem (nepovinné) / title after the name (optional)
  %Ustav / Department
  department={UITS}, % doplňte příslušnou zkratku dle ústavu na zadání: UPSY/UIFS/UITS/UPGM / fill in appropriate abbreviation of the department according to assignment: UPSY/UIFS/UITS/UPGM
  % Školitel / supervisor
  supervisor.name={Milan},   % jméno školitele / supervisor name 
  supervisor.surname={Češka},   % příjmení školitele / supervisor surname
  supervisor.title.p={RNDr.},   %titul před jménem (nepovinné) / title before the name (optional)
  supervisor.title.a={Ph.D.},    %titul za jménem (nepovinné) / title after the name (optional)
  % Klíčová slova / keywords
  keywords.cs={Sem budou zapsána jednotlivá klíčová slova v českém (slovenském) jazyce, oddělená čárkami.}, % klíčová slova v českém či slovenském jazyce / keywords in czech or slovak language
  keywords.en={Sem budou zapsána jednotlivá klíčová slova v anglickém jazyce, oddělená čárkami.}, % klíčová slova v anglickém jazyce / keywords in english
  %keywords.en={Here, individual keywords separated by commas will be written in English.},
  % Abstrakt / Abstract
  abstract.cs={Do tohoto odstavce bude zapsán výtah (abstrakt) práce v českém (slovenském) jazyce.}, % abstrakt v českém či slovenském jazyce / abstract in czech or slovak language
  abstract.en={Do tohoto odstavce bude zapsán výtah (abstrakt) práce v anglickém jazyce.}, % abstrakt v anglickém jazyce / abstract in english
  %abstract.en={An abstract of the work in English will be written in this paragraph.},
  % Prohlášení (u anglicky psané práce anglicky, u slovensky psané práce slovensky) / Declaration (for thesis in english should be in english)
  declaration={Prohlašuji, že jsem tuto bakalářskou práci vypracoval samostatně pod vedením pana X...
Další informace mi poskytli...
Uvedl jsem všechny literární prameny, publikace a další zdroje, ze kterých jsem čerpal.},
  %declaration={I hereby declare that this Bachelor's thesis was prepared as an original work by the author under the supervision of Mr. X
% The supplementary information was provided by Mr. Y
% I have listed all the literary sources, publications and other sources, which were used during the preparation of this thesis.},
  % Poděkování (nepovinné, nejlépe v jazyce práce) / Acknowledgement (optional, ideally in the language of the thesis)
  acknowledgment={V této sekci je možno uvést poděkování vedoucímu práce a těm, kteří poskytli odbornou pomoc
(externí zadavatel, konzultant apod.).},
  %acknowledgment={Here it is possible to express thanks to the supervisor and to the people which provided professional help
%(external submitter, consultant, etc.).},
  % Rozšířený abstrakt (cca 3 normostrany) - lze definovat zde nebo níže / Extended abstract (approximately 3 standard pages) - can be defined here or below
  %extendedabstract={Do tohoto odstavce bude zapsán rozšířený výtah (abstrakt) práce v českém (slovenském) jazyce.},
  %faculty={FIT}, % FIT/FEKT/FSI/FA/FCH/FP/FAST/FAVU/USI/DEF
  faculty.cs={Fakulta informačních technologií}, % Fakulta v češtině - pro využití této položky výše zvolte fakultu DEF / Faculty in Czech - for use of this entry select DEF above
  faculty.en={Faculty of Information Technology}, % Fakulta v angličtině - pro využití této položky výše zvolte fakultu DEF / Faculty in English - for use of this entry select DEF above
%   department.cs={Ústav matematiky}, % Ústav v češtině - pro využití této položky výše zvolte ústav DEF nebo jej zakomentujte / Department in Czech - for use of this entry select DEF above or comment it out
%   department.en={Institute of Mathematics} % Ústav v angličtině - pro využití této položky výše zvolte ústav DEF nebo jej zakomentujte / Department in English - for use of this entry select DEF above or comment it out
}

% Rozšířený abstrakt (cca 3 normostrany) - lze definovat zde nebo výše / Extended abstract (approximately 3 standard pages) - can be defined here or above
%\extendedabstract{Do tohoto odstavce bude zapsán výtah (abstrakt) práce v českém (slovenském) jazyce.}

% nastavení délky bloku s titulkem pro úpravu zalomení řádku - lze definovat zde nebo výše / setting the length of a block with a thesis title for adjusting a line break - can be defined here or above
%\titlelength{14.5cm}
% nastavení délky bloku s druhým titulkem pro úpravu zalomení řádku - lze definovat zde nebo výše / setting the length of a block with a second thesis title for adjusting a line break - can be defined here or above
%\sectitlelength{14.5cm}

% řeší první/poslední řádek odstavce na předchozí/následující stránce
% solves first/last row of the paragraph on the previous/next page
\clubpenalty=10000
\widowpenalty=10000

% checklist
\newlist{checklist}{itemize}{1}
\setlist[checklist]{label=$\square$}

\begin{document}
  % Vysazeni titulnich stran / Typesetting of the title pages
  % ----------------------------------------------
  \maketitle
  % Obsah
  % ----------------------------------------------
  \setlength{\parskip}{0pt}

  {\hypersetup{hidelinks}\tableofcontents}
  
  % Seznam obrazku a tabulek (pokud prace obsahuje velke mnozstvi obrazku, tak se to hodi)
  % List of figures and list of tables (if the thesis contains a lot of pictures, it is good)
  \ifczech
    \renewcommand\listfigurename{Seznam obrázků}
  \fi
  \ifslovak
    \renewcommand\listfigurename{Zoznam obrázkov}
  \fi
  % {\hypersetup{hidelinks}\listoffigures}
  
  \ifczech
    \renewcommand\listtablename{Seznam tabulek}
  \fi
  \ifslovak
    \renewcommand\listtablename{Zoznam tabuliek}
  \fi
  % {\hypersetup{hidelinks}\listoftables}

  \ifODSAZ
    \setlength{\parskip}{0.5\bigskipamount}
  \else
    \setlength{\parskip}{0pt}
  \fi

  % vynechani stranky v oboustrannem rezimu
  % Skip the page in the two-sided mode
  \iftwoside
    \cleardoublepage
  \fi

  % Text prace / Thesis text
  % ----------------------------------------------
  \ifenglish
    \chapter{Introduction}

Randomisation is essential to research areas such as \textit{probabilistic programming},~dependability (system components with uncertainty),~distributed computing (symmetry breaking),~and planning (unknown and noisy environments).
Probabilistic programs are powerful modelling apparatus of systems containing probabilistic uncertainty.
Systems with unreliable and unpredictable behaviour require the~use of such a~mathematical apparatus established on probability theory.
Designing such systems exhibiting a~desirable behaviour \,--\, e.g. selecting the~optimal power management strategy or a~network protocol increasing the~packet throughput \,--\, is challenging for reasoning over multiple alternative designs.
Their applications cover a~broad range of research areas,~including,~e.g. analysis of (quantitative) software product lines~\cite{spl1,spl2},~strategy synthesis in planning under partial observability~\cite{pomdp1,pomdp2},~or design of communication protocols~\cite{herman1,herman2}.

A~set of declarative temporal constraints often expresses the~efficiency and correctness of the~probabilistic programs.
The~model checkers for probabilistic systems,~such as \storm{}~\cite{STORM} or \prism{}~\cite{KNP11},~provide automated verification of such constraints.
However,~these probabilistic model checkers require a~fixed model or a~fixed program,~contrary to their usage requirements when modelling probabilistic programs.
Developers need to verify the~system designs as early as possible at the~developing process to maintain its costs tractable and as best as possible.
System designs are prevailingly incomplete at the~initial development phases because,~in most cases,~there are no known all system specifications or intentionally left out potentially.
These undefined system specifications are called \textit{holes},~and they can,~e.g.,~reflect an~unspecified component for wireless specification or a~partially implemented controller.
The~synthesis's primary purpose is through analysis reveals a~concrete subsystem with fully-defined behaviour and eventually reveals optimal designs when they are requested.
A~vital aspect of the design cycle is design space explorations,~i.e. exploring all possible designs.
When considering a~Markov chain (MC) as the~mathematical apparatus of a~probabilistic program,~then the~design space represents a~family of such chains,~and the~synthesis task is to find the~one that satisfies a~given specifications.

A~\textit{one-by-one} approach can naively solve the~synthesis problem by analysing all unique designs~\cite{spl3,onebyone}.
On the contrary,~the whole design space can also be modelled as a~single \textit{all-in-one} Markov decision process~\cite{spl3,allinone}.
However,~enumerating all members of design space (realisations) is unfeasible due to its combinatorial explosion, and the size of such all-in-one MDP is proportional to the number of candidate designs.
Unfortunately,~the~double state-space explosion problem renders both of these approaches infeasible for large families.
Other approaches consider evolutionary search algorithms for the~synthesis of software systems~\cite{spl2}.
These methods remain incomplete and cannot efficiently solve more challenging systems,~e.g.,~which design satisfies the specification \textit{optimally}.

In this work,~we will focus on a~complete state-of-the-art approach for the~synthesis of probabilistic programs.
This approach was first introduced in \textit{Andriushchenko} master’s thesis~\cite{roman-DP}, followed~by its improvement in~\cite{tacas21}.
It combines two sophisticated methods providing an~analysis of whole design sub-families at once.
The~first method analyses each design from a~given sub-family individually and constructs critical sub-systems of counter-examples to prune all designs behaving incorrectly \,--\, the~so-called \textit{counter-example guided inductive synthesis} (CEGIS)~\cite{cegis}.
The~second method,~called abstraction refinement (AR)~\cite{cegar},~immediately analyses the~entire design space and refines it into design sub-families when the~analysis yields inconclusive results.
Both of these methods have shown convincing results,~although each faces certain limitations.
They are incomparable because one method can be more suitable for specific probabilistic programs classes and conversely.
The~approach presented in this work integrates both these methods and it manages to significantly outperform them,~sometimes by a~margin of orders of magnitude.

All these presented methods consider \textit{topology synthesis} task assuming a~finite set of parameters that affect the~model topology,~where the~individual parameters represent the~undefined system specifications.
This task focuses on Markov chains families having different topologies of the~state space and,~as a~consequence,~different sets of reachable states.
However,~another area of synthesis tasks considers a~Markov chain with fixed topology but undefined transition probabilities.
This area has been discussed by approaches of model repairing~\cite{model-repair-1,pathak-et-al-nfm-2015} and techniques for \textit{parameter synthesis}~\cite{ceska2014robustness,Quatmann2016}.
When modelling real-world systems,~combining the two tasks can quickly occur,~but the~support to solve such a~\textit{combined synthesis} task does not exist.

\subsubsection*{Key Contributions.}
This thesis considers a~novel integrated method introduced in the~previous works~\cite{roman-DP,tacas21} as a~base stone.
Initially,~this method was designed only for feasibility synthesis task with one specification.
However,~probabilistic programs often have to satisfy specifications expressed as a~conjunction of several temporal logic constraints.
Therefore, we designed an~extension of this method to support \textit{multi-property specifications} and \textit{optimal synthesis} task.
The~designed extension for multi-properties is performed in the~same loop as the~origin single-property synthesis, with~necessary modifications: \textit{AR} need to analyse satisfiable specifications within inference sub-families no longer,~and \textit{CEGIS} analyses each specification individually and constructs counter-examples whenever a~given specification is unsatisfiable.
Consequently,~the~novel integrated method inherits the~benefits of \textit{AR} and \textit{CEGIS} in its favour also at multi-property synthesis.
\textit{Optimal synthesis} is a~particular instance of \textit{multi-property synthesis},~and it can find an application in various domains.
In particular,~specification set includes so-called violation property representing the currently optimal solution,~and its threshold is updated whenever a~new optimal solution is found.
Moreover,~we designed support for the~relaxed variant of the~optimal synthesis,~so-called $\varepsilon$-optimal synthesis,~which is in most cases even faster.
We evaluate designed extensions on an~extensive set of real-world case studies. 
We confirm the~results of a~novel approach presented in the~previous works and found the~following conclusions relating to these extensions.
A~novel integrated method is orders of magnitude faster than one-by-one enumeration when analysing the~single property.
A~multi-property synthesis slows down both approaches,~although the~novel method slow down is almost negligible.
An~optimal synthesis slows down only the~integrated approach,~yet it is still incomparably faster than enumeration.
The~assumption that $\varepsilon$-optimal synthesis can significantly speed up the~whole optimal synthesis process was also confirmed.

\subsubsection*{Structure of this paper.}


\chapter{Synthesis of Probabilistic Programs}


\chapter{Advanced Methods for Probabilistic Synthesis}


\chapter{Combined Probabilistic Synthesis}


\chapter{Tool Architecture}


\chapter{Experimental Evaluation}


\chapter{Final Considerations}


\section{Future Research}
\section{Conclusions}
  \else
    \input{projekt-01-kapitoly-chapters}
  \fi
  
  % Kompilace po částech (viz výše, nutno odkomentovat)
  % Compilation piecewise (see above, it is necessary to uncomment it)
  %\subfile{projekt-01-uvod-introduction}
  % ...
  %\subfile{chapters/projekt-05-conclusion}


  % Pouzita literatura / Bibliography
  % ----------------------------------------------
\ifslovak
  \makeatletter
  \def\@openbib@code{\addcontentsline{toc}{chapter}{Literatúra}}
  \makeatother
  \bibliographystyle{bib-styles/Pysny/skplain}
\else
  \ifczech
    \makeatletter
    \def\@openbib@code{\addcontentsline{toc}{chapter}{Literatura}}
    \makeatother
    \bibliographystyle{bib-styles/Pysny/czplain}
  \else 
    \makeatletter
    \def\@openbib@code{\addcontentsline{toc}{chapter}{Bibliography}}
    \makeatother
    \bibliographystyle{bib-styles/Pysny/enplain}
  %  \bibliographystyle{alpha}
  \fi
\fi
  \begin{flushleft}
  \bibliography{projekt-20-literatura-bibliography}
  \end{flushleft}

  % vynechani stranky v oboustrannem rezimu
  % Skip the page in the two-sided mode
  \iftwoside
    \cleardoublepage
  \fi

  % Prilohy / Appendices
  % ---------------------------------------------
  \appendix
\ifczech
  \renewcommand{\appendixpagename}{Přílohy}
  \renewcommand{\appendixtocname}{Přílohy}
  \renewcommand{\appendixname}{Příloha}
\fi
\ifslovak
  \renewcommand{\appendixpagename}{Prílohy}
  \renewcommand{\appendixtocname}{Prílohy}
  \renewcommand{\appendixname}{Príloha}
\fi
%  \appendixpage

% vynechani stranky v oboustrannem rezimu
% Skip the page in the two-sided mode
%\iftwoside
%  \cleardoublepage
%\fi
  
\ifslovak
%  \section*{Zoznam príloh}
%  \addcontentsline{toc}{section}{Zoznam príloh}
\else
  \ifczech
%    \section*{Seznam příloh}
%    \addcontentsline{toc}{section}{Seznam příloh}
  \else
%    \section*{List of Appendices}
%    \addcontentsline{toc}{section}{List of Appendices}
  \fi
\fi
  \startcontents[chapters]
  \setlength{\parskip}{0pt} 
  % seznam příloh / list of appendices
  % \printcontents[chapters]{l}{0}{\setcounter{tocdepth}{2}}
  
  \ifODSAZ
    \setlength{\parskip}{0.5\bigskipamount}
  \else
    \setlength{\parskip}{0pt}
  \fi
  
  % vynechani stranky v oboustrannem rezimu
  \iftwoside
    \cleardoublepage
  \fi
  
  % Přílohy / Appendices
  \ifenglish
    \chapter{Storage Medium}
\begin{itemize}
   \item[] \texttt{/synthesis/*}\,---\,source code of \toolname{} from date May 25, 2021
   \item[] \texttt{/README.txt}\,---\,useful information about the storage medium content
   \item[] \texttt{/text/*}\,---\,source code of this thesis
   \item[] \texttt{/xstupi00.pdf}\,---\,final version of this thesis
\end{itemize}
  \else
    \input{projekt-30-prilohy-appendices}
  \fi
  
  % Kompilace po částech (viz výše, nutno odkomentovat)
  % Compilation piecewise (see above, it is necessary to uncomment it)
  %\subfile{projekt-30-prilohy-appendices}
  
\end{document}
